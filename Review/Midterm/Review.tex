\documentclass[12pt]{article} 
\usepackage[utf8]{inputenc}
\usepackage{geometry}
\geometry{letterpaper}
\usepackage{graphicx} 
\usepackage{parskip}
\usepackage{booktabs}
\usepackage{array} 
\usepackage{paralist} 
\usepackage{verbatim}
\usepackage{subfig}
\usepackage{fancyhdr}
\usepackage{sectsty}
\usepackage[shortlabels]{enumitem}

\pagestyle{fancy}
\renewcommand{\headrulewidth}{0pt} 
\lhead{}\chead{}\rhead{}
\lfoot{}\cfoot{\thepage}\rfoot{}

\geometry{
    left=0.25in,
    right=0.25in,
    top=0.25in,
    bottom=0.25in,
}

%%% ToC (table of contents) APPEARANCE
\usepackage[nottoc,notlof,notlot]{tocbibind} 
\usepackage[titles,subfigure]{tocloft}
\renewcommand{\cftsecfont}{\rmfamily\mdseries\upshape}
\renewcommand{\cftsecpagefont}{\rmfamily\mdseries\upshape} %

\usepackage{amsmath}
\usepackage{amssymb}
\usepackage{mathtools}
\usepackage{empheq}
\usepackage{xcolor}

\usepackage{tikz}
\usepackage{pgfplots}
\usepackage{tikz-cd}
\pgfplotsset{compat=1.18}

\newcommand{\ans}[1]{\boxed{\text{#1}}}
\newcommand{\vecs}[1]{\langle #1\rangle}
\renewcommand{\hat}[1]{\widehat{#1}}

\newcommand{\F}[1]{\mathcal{F}(#1)}
\renewcommand{\P}{\mathbb{P}}
\newcommand{\R}{\mathbb{R}}
\newcommand{\E}{\mathbb{E}}
\newcommand{\Z}{\mathbb{Z}}
\newcommand{\C}{\mathbb{C}}
\newcommand{\N}{\mathbb{N}}
\newcommand{\Q}{\mathbb{Q}}

\newcommand{\ind}{\mathbbm{1}}
\newcommand{\qed}{\quad \blacksquare}

\newcommand{\brak}[1]{\left\langle #1 \right\rangle}
\newcommand{\bra}[1]{\left\langle #1 \right\vert}
\newcommand{\ket}[1]{\left\vert #1 \right\rangle}
\newcommand{\abs}[1]{\left\vert #1 \right\vert}
\newcommand{\mfX}{\mathfrak{X}}
\newcommand{\norm}[1]{\left\vert \left\vert #1 \right\vert \right\vert}

\newcommand{\SL}{\text{SL}\,}
\newcommand{\SO}{\text{SO}\,}
\newcommand{\GL}{\text{GL}\,}

\newcommand{\tr}{\text{tr}\,}

\newcommand{\biject}{\hookrightarrow \hspace{-8pt} \rightarrow}

\newcommand{\g}{\mathfrak{g}}
\renewcommand{\sf}{\mathfrak{s}}
\newcommand{\h}{\mathfrak{h}}
\renewcommand{\sl}{\mathfrak{sl}}
\newcommand{\gl}{\mathfrak{gl}}
\newcommand{\so}{\mathfrak{so}}

\newcommand{\im}{\text{im}\,}

\newcommand{\Ad}{\text{Ad}}
\newcommand{\ad}{\text{ad}}
\newcommand{\Aut}{\text{Aut}}

\usepackage{tcolorbox}
\tcbuselibrary{breakable, skins}
\tcbset{enhanced}
\newenvironment*{tbox}[2][gray]{
    \begin{tcolorbox}[
        parbox=false,
        colback=#1!5!white,
        colframe=#1!75!black,
        breakable,
        title={#2}
    ]}
    {\end{tcolorbox}}


\title{Math 1820A - Midterm Review}
\author{Milan Capoor}
\date{}

\begin{document}
\maketitle

\section{Lecture Notes} 

A matrix is \textbf{diagonalizable} if it is similar to a diagonal matrix, i.e. $A = PDP^{-1}$. 

\textbf{Theorem:} $A$ is diagonalizable if the algebraic multiplicity of each eigenvalue is equal to the geometric multiplicity of each eigenvalue. (The algebraic multiplicity of an eigenvalue is the number of times it appears as a root of the characteristic polynomial, and the geometric multiplicity is the dimension of the eigenspace corresponding to that eigenvalu - $\dim N(A - \lambda_i I)$.)

\textbf{Jordan-Chevalley Decomposition:} Each linear map decomposes as a sum of a diagonalizable and a nilpotent part where the two pieces commute. 

\textbf{Jordan Canonical Form:} up to similarity, every matricx in $\C^{n\times m}$ looks like 
\[\begin{pmatrix}
    J_1\\ 
    & \ddots\\
    && J_k
\end{pmatrix}, \quad J_i = \begin{pmatrix}
    \lambda_i & 1\\
    & \lambda_i & 1\\
    && \ddots & 1\\
    &&& \lambda_i
\end{pmatrix}\]

\textbf{Cayley-Hamilton Theorem:} For a linear map $A$, $p(A) = 0$ where $p(\lambda)$ is the characteristic polynomial of $A$.

A \textbf{one-parameter subgroup} is a homomorphism $\alpha: \R \to \GL(n, \R)$, i.e. $\alpha(t+s) = \alpha(t)\alpha(s)$ and $\alpha(0) = I_n$.

\textbf{Theorem:} If $A$ is a one-parameter subgroup of  $\GL(n, \C)$, then there exists a unique $X \in M_n(\C)$ such that $A(t) = e^{tX}$.

\textbf{Matrix Exponential:}
\[e^A = I_n + A + \frac{1}{2}A^2 + \frac{1}{3!}A^3 + \dots + \sum_{n=0}^{\infty} \frac{1}{n!}A^n\]

\textbf{Theorem:} Let $A, B$ be commuting matrices. Then $e^A e^B = e^{A+B}$

\textbf{Theorem:} $\det(e^A) = e^{\tr A}$ for any matrix $A \in M_n(\C)$

\emph{Consequence:} $\exp: M_n(\R) \to \GL(n, \R)$ lies in the identity component of $\GL(n, \R)$.

\textbf{Theorem:} Let $A(t)$ be a smooth family of matrices in $\GL(2, \R)$ and $A(0) = I$. Then 
\[\frac{d}{dt}\bigg\vert_{t=0} \det A(t) = \tr A'(0)\]

We say $G \subseteq \GL(n,\C)$ is a \textbf{matrix lie group} if it is a smooth submanifold of $\GL(n, \C)$, i.e. at every point $g \in G$, we can find a patch $U \subseteq \R^k \biject \phi(I) \subseteq G$. This allows us to do calculus on $G$.

Alternatively, $G$ is a \textbf{lie group} if it is a smooth manifold and a group whose multiplication and inverse operations are smooth maps. 

\textbf{Lie group dimenesions:}
\begin{itemize}
    \item $\GL(n, \R)$ iś $n^2$-dimensional 
    \item $\dim \sl(n, \R) = \dim \SL(n, \R) = n^2 = 1$
    \item $\dim \so(3, \R) = 3$
\end{itemize}

\textbf{Claim:} the dimension of the lie group is equal to the dimension of the lie algebra because the exponential map is a local diffeomorphism. 

\textbf{Definition:} A vector space with a pairing $[\cdot, \cdot]: \g \times \g \to \g$ is a \textbf{lie algebra} iff it is: 
\begin{enumerate}
    \item Skew-symmetric (${B, A} = -[A, B]$)
    \item Bilinear in both variables ($[A + cB, D] = [A, D] + c[B, D]$)
    \item Satisfies the Jacobi identity ($[A, [B, C]] + [B, [C, A]] + [C, [A, B]] = 0$)
\end{enumerate}

We define this bracket operation by 
\[[A, B] = \frac{d}{dt}\bigg\vert_{t=0} \frac{d}{ds}\bigg\vert_{s=0} e^{tA}e^{sB}e^{-tA} = AB - BA\]

In other words, this is a pairing motivated by conjugation which takes tangent vectors to other tangent vectors. Further, because of the ambient manifold, we can identify tangent vectors with elements of the group. 

Thus, we can think of the bracket as a function 
\begin{align*}
    G &\to \Aut(G)\\ 
    G&\mapsto ghg^{-1}
\end{align*}

Alternatively, the lie algebra can be defined as the set 
\[\g = \{X: e^{tX} \in G \; \forall t \in \R\}\]

\textbf{Canonical Lie Algebras:}
\begin{itemize}
    \item $G = \R^n \simeq \text{Diag}(n, \R) \implies [\g, \g] = 0$ 
    \item $G = \text{Aff}^+(1, \R) = \left\{\begin{pmatrix}
        a & b\\ 
        0 & 1
    \end{pmatrix} \bigg\vert a >0, b \in \R\right\} = \R X \oplus \R Y$ has lie algebra 
    \[\g = \begin{pmatrix}
        c & d\\ 
        0 & 0
    \end{pmatrix}, \quad [X, Y] = Y\]
    \item $G = \SL(2, \R)$ has lie algebra 
    \[\g = \begin{pmatrix}
        a & b\\ 
        c & -a
    \end{pmatrix}, \quad \begin{array}{l}
        \mbox{$[X, Y] = Z$}\\ 
        \mbox{$[X, Z] = 2X$}\\ 
        \mbox{$[Z, Y] = -2Y$}
    \end{array}\]
    \item $G = \SO(3, \R)$ has lie algebra
    \[\g = \so(3, \R) = \{B \in M_3(\R) : B + B^T = 0\} = \begin{pmatrix}
        0 & -c & b\\ 
        c & 0 & -a\\ 
        -b & a & 0
    \end{pmatrix}, \quad \begin{array}{l}
        \mbox{$[A, B] = C$}\\ 
        \mbox{$[B, C] = A$}\\ 
        \mbox{$[C, A] = B$}
    \end{array}\]
    \item $\mathfrak{su}(n) = \{X \in M_n(\C): X + X^* = 0, \tr X = 0\}$ 
    \item $\sl(n, \R) = \{X \in M_n(\R): \tr X = 0\}$
    \item $\gl(n, \R) = M_n(\R)$
    \item The Heisenberg group $G = \begin{pmatrix}
        1 & c & a\\ 
        0 & 1 & b\\ 
        0 & 0 & 1
    \end{pmatrix}$ has lie algebra 
    \[\g = \begin{pmatrix}
        0 & c& a\\ 
        0 & 0 & b\\ 
        0 & 0 & 0
    \end{pmatrix}, \quad \begin{array}{c}
        \mbox{$[A, B] = C$}\\ 
        \mbox{$[A, C] = 0$}\\
        \mbox{$[B, C] = 0$}
    \end{array}\]
    
\end{itemize}

The set of matrices $g \in G$ such that there is a path $\gamma: [0, 1] \to G$ from $1 \in G$ to $g \in G$ is called the \textbf{identity component} of $G$.

\textbf{Claim:} The identity component of a lie group is normal in $G$. 

A vector subspace $\mathfrak{s} \subseteq \g$ is a \textbf{subalgebra} if is is closed under lie brackets: $[\mathfrak{s}, \mathfrak{s}] \subseteq \mathfrak{s}$.

\textbf{Theorem:} There is a correspondence between sub-algebras of $\g$ and connected lie-subgroups of $G$. 
\[S \subseteq G \implies \mathfrak{s} \subseteq \g\]

A sub-algebra $\h$ of $\g$ is an \textbf{ideal} if $[\g, \h] \subseteq \h$. 

\textbf{Proposition:} If $H \triangleleft F$, we have a short exact sequence on the level of lie algebras, 
\[\begin{tikzcd}
    H \arrow[hook]{r} & G \arrow[two heads]{r} & G/H\\ 
    \h \arrow[hook]{r} & \g \arrow[no head]{u} \arrow[two heads]{r} & \g/\h
\end{tikzcd}\]

A \textbf{short exact sequence of groups} is a sequence of groups and group homomorphisms
\[N \overset{i}{\hookrightarrow} G \overset{p}{\twoheadrightarrow} Q\]
where $\ker p = \im i$, $i$ is injective, and $p$ is surjective. 

\textbf{Theorem:} Let $\phi: G \to H$ be a group homomorphism. Then the following diagram commutes:
\[\begin{tikzcd}
    G \arrow{r}{\phi} & H\\ 
    \g \arrow{u}{\exp} \arrow{r}{d\phi_0} & \h \arrow{u}{\exp}
\end{tikzcd}\]

A linear map $\psi: \g \to \h$ is a \textbf{Lie algebra homomorphism} if it preserves brackets: 
\[\psi([X, Y]) = [\psi(X), \psi(Y)]\]

A short exact sequence of groups which admits a section $\sigma: H \to G$ satisfying 
\[\begin{tikzcd}
    N \arrow[hook]{r}{i} & G \arrow[two heads]{r}{p} & H \arrow[bend right=60]{l}{\sigma}{\sigma}
\end{tikzcd}\]
with $p\sigma = 1_H$ is a \textbf{split extension}. 

In general, for any groups $G$ and $H$, we may define a split exact sequence by 
\[G \hookrightarrow G \times H \twoheadrightarrow H\]
where $S = G \times H = GH$ 

Given a homomorphism $\phi: H \to \Aut(N)$, we can form the \textbf{semidirect product} $G_{\phi} = N \rtimes_\phi H$ (with $N \triangleleft G$) by defining the group operation
\[(n, h)(a, b) = (n\phi_h(a), hb)\]
which is equivalent to defining a split exact sequence $N \hookrightarrow G_{\phi} \twoheadrightarrow H$.

\textbf{Claim:} $\R^2$ and $\mathfrak{aff}(1, \R)$ are the \emph{only} 2-d lie algebras. 

For a lie algebra $\g = \text{Span}\{X, Y, Z\}$, we say $X$ is \textbf{central} if 
\[X \in Z(G) = \{A \in \g: [A, \g] = 0\}\]

\textbf{Claim:} $Z(\g)$ is the lie algebra of the $Z(G)$. 

\textbf{Lie's Theorems:} 
\begin{enumerate}
    \item For any lie algebra $\g$, there is a simply connected lie group $G$ with lie algebra $\g$. 
    
    By relaxing the topological restraint, we can make the powerful statement: any lie algebra $\g$ can be realized as the lie algebra of a linear group 

    \item For any lie algebra homomorphism $\psi: \g \to \h$ where $\g, \h$ are lie algebras of $G, H$ and $G$ is simply connected, there exists a group homomorphism $\phi: G \to H$ such that $d\phi = \psi$. 
    
    \item The lie subalgebras of $\g$ correspond (bijectively) to the connected lie subgroups of $G$ 
\end{enumerate}

The \textbf{adjoint representation} $\Ad: G \to \Aut(\g)$ is given by $g \mapsto \{x \mapsto dI_g(X)\}$ where $I_g: G \biject G$ by $h \mapsto ghg^{-1}$. Then letting $X \sim e^{tX}$, 
\[\Ad_g(X) = \frac{d}{dt}\bigg\vert_{t=0} ge^{tX}g^{-1} = gXg^{-1}\]
so $\Ad_g: \g \biject \g$ is defined by $X \mapsto gXg^{-1}$. 

Further, each $\Ad_g$ is a lie-algebra automorphism 
\[\Ad_g({X, Y}) = [\Ad_g(X), \Ad_g(Y)]\]
since $I_g$ is a group automorphism.

Further, 
\[\frac{d}{dt}\bigg\vert_{t=0} \Ad_{g(t)} = \ad(g'(0)) = \ad(X)\] 
because we have the commutative diagram 
\[\begin{tikzcd}
    G \arrow{r}{\Ad} & \Aut(\g)\\ 
    \g \arrow{u}{\exp} \arrow{r}{\ad} & \gl(\g) \arrow{u}{\exp}
\end{tikzcd}\]

A \textbf{bilinear form} is a map $B: \g \times \g \to \R$ (or $\C$) which is 
\begin{enumerate}
    \item Symmetric: $B(X, Y) = B(Y, X)$
    \item Bilinear: $B(\cdot, X)$ and $B(X, \cdot)$ with $X$ fixed are linear maps $\g \to \R$ 
\end{enumerate}

If $B(X, Y) = 0 \forall Y \in \g \implies X = 0$, then $B$ is \textbf{non-degenerate}.

To each bilinear form, you may assign a symmetric matrix $A$ by $B(X, Y) = x^T Ay$. Its eigenvalues will all be real (by symmetric) and the signs of the eigenvalues are invariant with respect to basis. The collection of signs of eigenvalues is the \textbf{signature} of the bilinear form. 

\textbf{Proposition:} A bilinear form is non-degenerate iff its signature has no zeros. 

We say a bilinear form on $V$ is \textbf{$G$-invariant} if we have a $G$-action on $V$ that preserves the bilinear form 
\[g: V \biject V \implies B(gv, gw) = B(v, w)\] 

For each $X, Y \in \g$, the \textbf{Killing form} is given by 
\[B(X, Y) = \tr(\ad_X \circ \ad_Y)\]
where $\ad_X: \g \to \g$ is given by $\ad_X(Y) = [X, Y]$.

Further, the Killing form is Ad-invariant:
\[B(\Ad_g X, \Ad_g Y) = B(X, Y)\]
Therefore, $B$ is an invariant of the lie-algebra. \emph{Conclusion:} for two isomorphic lie algebras, the Killing forms are isometric. This provides a very simple way to show that two lie algebras are not isomorphic.

Given a lie algebra $\g = \{X, Y, Z\}$, we need to calculate the Killing form 
\[\begin{pmatrix}
    B(X, X) & B(X, Y) & B(X, Z)\\ 
    & B(Y, Y) & B(Y, Z)\\
    && B(Z, Z)
\end{pmatrix}\]

\textbf{Theorem:} Let $\g$ be a nilpotent lie-algebra. Then $K = 0$ on $\g$ 

\textbf{Cartan's Criteria for Semi-simplicity:} A lie algebra $\g$ is\textbf{ semi-simple} (has no nonzero solvable ideals) iff the Killing form is non-degenerate

A lie algebra is \textbf{solvable} iff the derived series 
\[\g^0 = g, \; \g^{n+1} = [\g^n, \g^n]\]
vanishes in finitely many steps. 

A group is solvable iff $G^1 = [G, G] = g^{-1}h^{-1}gh, G^{n+1} = [G^n, G^n]$ terminates in finite time. Equivalently, $G$ is solvable iff there exists a composition series
\[1 = G_n \triangleleft G_{n-1} \triangleleft \dots G_1 \triangleleft G_0 = G\]
where each $G_i/G_{i+1}$ is abelian. 

\textbf{Observation:} Any abelian group is solvable. 

\textbf{Levi Decomposition Theorem:} Every finite dimensional lie algebra fits into a split exact seuqence $\h \hookrightarrow \g \twoheadrightarrow \g/\h$ where $\h$ is the unique maximal solvable ideal of $\g$ and $\g/\h$ is semi-simple. 

\textbf{Theorem:} Any solvable group (over an algebraically closed field) can be embedded in the upper triangular matrices. 

\textbf{Lemmas:}
\begin{itemize}
    \item If $H \subseteq G$ with $G$ solvable, then $H$ is solvable 
    \item If $\phi: G \twoheadrightarrow H$ and $G$ is solvable then $H$ is solvable. 
    \item If $N \hookrightarrow G \twoheadrightarrow H$, $G$ is solvable iff both $N$ and $H$ are solvable. 
\end{itemize}

A group is \textbf{nilpotent} iff the lower central series 
\[G_0 = G, \; G_{n+1} = [G, G_n]\]
converges to $1$ in finite time. 

Further, each $G_n$ is normal in $G$ which gives the composition series 
\[1 = G_n \triangleleft G_{n-1} \triangleleft \dots \triangleleft G_1 \triangleleft G_0 = G\]
so we can always form the quotient $G/G_i$. 

\textbf{Note:} Nilpotent implies solvable but solvable does not imply nilpotent. 

\pagebreak
\section{Homework Results}

Let $\alpha: \R \to \GL(n, \R)$ be a one parameter subgroup. then $\alpha'(t) = \alpha(t)\alpha'(0)$. Further, if $\alpha(0) = 1$ and $\alpha'(0) = A$, then $\alpha(t)$ must commute with $A$ for all $t$. 

If $A$ is any matrix in $M_n(\C)$, there exists a sequence of diagonalizable matrices $(A_n) \to A$. 

We define the lie algebra of $G \subset \GL(n, \C)$ to be  
\[\g = \{B \in M_n: B = \gamma'(0) \text{ where } \gamma \text{ is a smooth path in G satisfying } \gamma(0) = 1\}\]
Then, there exists a one-to-one correspondence between one parameter subgroups of $G$ and elements of $\g$. Thus, \emph{every one parameter subgroup is exponential.}

Let $N$ and $H$ be groups. Let $\phi: H \to \Aut(N)$ be a group homomorphism. Then the \textbf{semi-direct product} $G = N \rtimes_{\phi} H$ is the set $N \times H$ with the group operation 
\[(n, h)(a, b) = (n\phi_h(a), hb)\]

The \textbf{center} of a lie algebra is
\[Z(\g) = \{X \in \g: [X, \g] = 0\}\]

Every two dimensional non-abelian lie algebra $\g$ is isomorphic to $\mathfrak{aff}(1, \R) = \brak{X, Y}$ with $[Y, X] = X$

Let $\g$ be a Lie algebra with ideal $\h$. $\g/\h$ is abelian iff $[\g, \g] \subseteq \h$ 

Let $\g$ be the lie algebra of a connected lie group $G$. If $\g$ is abelian, $G$ is abelian. 

$\exp: \g \to G$ is surjective if $\g$ is abelian. 

\pagebreak
\section{Essential Formulae}

\textbf{One-parameter subgroup:}
\begin{enumerate}
    \item $\alpha(t + s) = \alpha(t)\sigma(s)$
    \item $\alpha(0) = I$
    \item $\alpha'(t) = \alpha(t)\alpha'(0)$
\end{enumerate}

\textbf{Matrix Exponential:}
\[e^{tA} = \sum_{n=1}^{\infty} \frac{t^n}{n!}A^n = I_n + tA + \frac{t^2}{2}A^2 + \frac{t^3}{3!}A^3 + \dots\]

If $AB = BA$, then
\[e^A e^B = e^{A + B}\]

For any matrix $A \in M_n(\C)$, 
\[\det(e^A) = e^{\tr A}\] 

\textbf{Lie Algebra:} a vector space with a paring $[\cdot, \cdot]: \g \times \g \to \g$
\begin{enumerate}
    \item $[A, B] = -[B, A]$ 
    \item $[A + cB, D] = [A, D] + c[B, D]$
    \item $[A, [B, C]] + [B, [C, A]] + [C, [A, B]] = 0$  
\end{enumerate}
defined by 
\[\ad_A(B) = [A, B] = \frac{d}{dt}\bigg\vert_{t=0}\, \frac{d}{ds}\bigg\vert_{s=0} e^{tA}e^{sB}e^{-tA} = AB - BA \]

A lie algebra is also 
\begin{enumerate}
    \item a vector space with a bracket operation
    \item the tangent space at the identity of a Lie group 
    \item $\g = \{X: e^{tX} \in G \quad \forall t \in \R\}$
\end{enumerate}

For $H \triangleleft G$, 
\[\begin{tikzcd}
    H \arrow[hook]{r} & G \arrow[two heads]{r} & G/H\\ 
    \h \arrow[hook]{r} \arrow{u}{\exp} & \g \arrow{u}{\exp} \arrow[two heads]{r} & \g/\h \arrow{u}{\exp}
\end{tikzcd}\]

A lie algebra homomorphism preserves brackets 
\[\psi([X, Y]) = [\psi(X), \psi(Y)]\]

A split exact sequence of groups 
\[\begin{tikzcd}
    N \arrow[hook]{r}{i} & G \arrow[two heads]{r}{p} & H \arrow[bend right=60]{l}{\sigma}
\end{tikzcd}\]
satisfies $\im i = \ker p$ (short) and $p\sigma = 1$ 

A semidirect product $G_{\phi} = N \rtimes_{\phi} H$ is the set $N \times H$ equipped with the group operation
\[(n, h)(a, b) = (n\phi_h(a), hb)\]
for homomorphism $\phi: H \to \Aut(N)$

\textbf{Representations:} 
\begin{itemize}
    \item $\Ad: G \to \Aut(\g)$ with $\Ad_g(X) = gXg^{-1}$
    \item $\ad_X: \g \to \g$ with $\ad_X(Y) = [X, Y]$
\end{itemize}
forming the diagram 
\[\begin{tikzcd}
    G \arrow{r}{\Ad} & \Aut(\g)\\ 
    \g \arrow{u}{\exp} \arrow{r}{\ad} & \gl(\g) \arrow{u}{\exp}
\end{tikzcd}\]

A \textbf{bilinear form} $B: \g \times \g \to \R$ is 
\begin{enumerate}
    \item Symmetric: $B(X, Y) = B(Y, X)$
    \item Bilinear
\end{enumerate}

If the signature has no zeros, $B$ is non-degenerate.

\textbf{Killing Form:}
\[\begin{pmatrix}
    B(X, X) & B(X, Y) & B(X, Z)\\ 
    & B(Y, Y) & B(Y, Z)\\
    && B(Z, Z)
\end{pmatrix} = \begin{pmatrix}
    \tr(\ad_X \circ \ad_X) & \tr(\ad_X \circ \ad_Y) & \tr(\ad_X \circ \ad_Z)\\ 
    & \tr(\ad_Y \circ \ad_Y) & \tr(\ad_Y \circ \ad_Z)\\
    && \tr(\ad_Z \circ \ad_Z)
\end{pmatrix}\]

$K = 0$ for a nilpotent lie algebra. $K$ is non-degenerate for a semi-simple lie algebra.

\textbf{Solvable:} the derived series 
\[\g^0 = \g, \; \g^n = [\g^{n-1}, \g^{n-1}]\]
terminates. 

$G_n/[G_{n-1}, G_{n-1}]$ is abelian.

\textbf{Nilpotent:} the lower central series 
\[\g_0 = \g, \; \g_n = [\g, \g_{n-1}]\]

$G/[G, G_{n-1}]$ is normal. 

\[\text{Nilpotent} \implies \text{Solvable}\]

\begin{itemize}
    \item Subgroups are solvable if group is solvable 
    \item $N \hookrightarrow G \twoheadrightarrow H$, $G$ is solvable iff both $N$ and $H$ are solvable.
\end{itemize}

$\h$ is ideal if $[X, \h] \subseteq \h$. $\g/\h$ is abelian iff $[\g, \g] \subseteq \h$. 

For a connected $G$, $\g$ abelian $\implies G$ abelian. $\g$ abelian implies $\exp$ is surjective. 

\end{document}