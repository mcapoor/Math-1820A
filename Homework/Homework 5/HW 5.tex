\documentclass[12pt]{article} 
\usepackage[utf8]{inputenc}
\usepackage{geometry}
\geometry{letterpaper}
\usepackage{graphicx} 
\usepackage{parskip}
\usepackage{booktabs}
\usepackage{array} 
\usepackage{paralist} 
\usepackage{verbatim}
\usepackage{subfig}
\usepackage{fancyhdr}
\usepackage{sectsty}
\usepackage[shortlabels]{enumitem}

\pagestyle{fancy}
\renewcommand{\headrulewidth}{0pt} 
\lhead{}\chead{}\rhead{}
\lfoot{}\cfoot{\thepage}\rfoot{}

\geometry{
    left=20mm,
    right=20mm,
    top=20mm,
    bottom=20mm,
}

%%% ToC (table of contents) APPEARANCE
\usepackage[nottoc,notlof,notlot]{tocbibind} 
\usepackage[titles,subfigure]{tocloft}
\renewcommand{\cftsecfont}{\rmfamily\mdseries\upshape}
\renewcommand{\cftsecpagefont}{\rmfamily\mdseries\upshape} %

\usepackage{amsmath}
\usepackage{amssymb}
\usepackage{mathtools}
\usepackage{empheq}
\usepackage{xcolor}

\usepackage{tikz}
\usepackage{pgfplots}
\pgfplotsset{compat=1.18}

\newcommand{\ans}[1]{\boxed{\text{#1}}}
\newcommand{\vecs}[1]{\langle #1\rangle}
\renewcommand{\hat}[1]{\widehat{#1}}
\newcommand{\F}[1]{\mathcal{F}(#1)}
\renewcommand{\P}{\mathbb{P}}
\newcommand{\R}{\mathbb{R}}
\newcommand{\E}{\mathbb{E}}
\newcommand{\Z}{\mathbb{Z}}
\newcommand{\N}{\mathbb{N}}
\newcommand{\Q}{\mathbb{Q}}
\newcommand{\ind}{\mathbbm{1}}
\newcommand{\qed}{\quad \blacksquare}
\newcommand{\brak}[1]{\left\langle #1 \right\rangle}
\newcommand{\bra}[1]{\left\langle #1 \right\vert}
\newcommand{\ket}[1]{\left\vert #1 \right\rangle}
\newcommand{\abs}[1]{\left\vert #1 \right\vert}
\newcommand{\mfX}{\mathfrak{X}}
\newcommand{\ep}{\varepsilon}

\newcommand{\gl}{\mathfrak{gl}}
\newcommand{\GL}{\text{GL}\,}
\newcommand{\SL}{\text{SL}\,}
\newcommand{\sll}{\mathfrak{sl}}
\newcommand{\pgl}{\mathfrak{pgl}}
\newcommand{\g}{\mathfrak{g}}
\newcommand{\h}{\mathfrak{h}}
\newcommand{\im}{\text{im}\,}

\usepackage{tcolorbox}
\tcbuselibrary{breakable, skins}
\tcbset{enhanced}
\newenvironment*{tbox}[2][gray]{
    \begin{tcolorbox}[
        parbox=false,
        colback=#1!5!white,
        colframe=#1!75!black,
        breakable,
        title={#2}
    ]}
    {\end{tcolorbox}}


\title{Math 1820A: Homework 5}
\author{Milan Capoor}
\date{}

\begin{document}
\maketitle

\section*{Problem 1}
Finish the proof we started in class that every two dimensional non-abelian Lie algebra $\g$ is isomorphic to $\mathfrak{aff}(1,\R)$, the Lie-algebra generated by $X$ and $Y$ defined by $[Y,X] = X$.

    \color{blue}
        Let $\g$ and $\h$ be non-commutative 2-dimensional algebras. We seek to show that $\g \simeq \h \simeq \mathfrak{aff}(1,\R)$. 

        It suffices to show that $\g \simeq \mathfrak{aff}(1,\R) = \R X \oplus \R Y$ with $[Y, X] = X$. Let $\g = \R A \oplus \R B$. Suppose $[A, B] = xA + yB$. Since $\g$ is not commutative, $x$ and $y$ are not both zero. 

        \color{blue}
        WLOG, suppose $y = 0$. Then $[A, B] = xA$ for $x \neq 0$ so $[A, B/x] = A$. By symmetry, $[B/x, A] = -A$. 
        We make the substitutions $A' = A, B' = -B/x$ and define $\phi: \begin{array}{c}
            A' \mapsto X \\
            B' \mapsto Y
        \end{array}$. 

        Then 
        \[[B', A'] = [-\frac{B}{x}, A] = A = A'\]
        and we are done. 

        Now we check the case $y \neq 0$, i.e. $[A, B] = xA + yB$.

        We make the substitutions $A' = A + \frac{y}{x}B$, $B' = -\frac{B}{x}$ and define $\phi: \begin{array}{c}
            A' \mapsto X\\ 
            B' \mapsto Y
        \end{array}$ 
        so 
        \begin{align*}
            [X, Y] \sim [A', B'] &= [A + \frac{y}{x}B, -\frac{B}{x}]\\
                &= [A, -\frac{B}{x}] + [\frac{y}{x}B, -\frac{B}{x}]\\
                &= -\frac{1}{x}[A, B] - \frac{y}{x^2}[B, B]\\ 
                &= -\frac{1}{x}(xA + yB)\\
                &= -A - \frac{y}{x}B\\ 
                &= -A'\\
            [Y, X] \sim [B', A'] = A'
        \end{align*}
        and we are done. $\qed$

        \color{black}
\pagebreak

\section*{Problem 2} 
Let $\g$ be a Lie-algebra and $\h$ an ideal. Prove that $\g/\h$ is abelian if and only if $\h$ contains the commutator ideal $[\g,\g]$ of $\g$.  

    \color{blue}
        We want to show that $\g/\h \simeq \R^n$ iff $[\g, \g] \in \h$.  

        If $\g/\h$ is abelian, then $[\g/\h, \g/\h] = 0$. Let $X, Y \in \g$ so we can write 
        \begin{align*}
            [\g/\h, \g/\h] &= [X + \h, Y + \h]\\ 
                &= [X, Y] + [\h, Y] + [X, \h] + [\h, \h]\\ 
                &= [X, Y] - [Y, \h] + [X, \h] + [\h, \h]\\ 
                &= [X, Y] + [\h, \h]
        \end{align*}
    
        Then if $[\g/\h, \g/\h] = 0$, we must have that $[X, Y] = [\g, \g] \in \h$.

        Conversely, if $[\g, \g] \in \h$, then for all $X, Y \in \g$, we have that $[X, Y] \in \h$. Then
        \begin{align*}
            [\g/\h, \g/\h] &= [X + \h, Y + \h]\\ 
                &= [X, Y] + [\h, Y] + [X, \h] + [\h, \h]\\ 
                &= [X, Y] - [Y, \h] + [X, \h] + [\h, \h]\\ 
                &= [X, Y] + [\h, \h]\\ 
                &= [\h, \h]
        \end{align*}
        But this implies that $X + \h, Y + \h \in \h \implies X, Y \in \h$ so $\g = \h$. 

        Let $g \in \g$ and $h \in \h$. Since $g \in \h$ too,  
        \[ghg^{-1}h^{-1} \in \h\] 
        so $\g/\h$ is abelian. $\qed$
    \color{black}
\pagebreak


\section*{Problem 3}
Let $\g$ be the Lie-algebra of a connected Lie-group $G$. If $\g$ is abelian, prove that $G$ is abelian.  (You may use the fact that $\exp(\g) \subset G$ generates $G$ as a group.)

    \color{blue} 
        Let $\g$ be the abelian lie algebra of a connected lie group $G$. Let $A, B \in G$. 

        Since $\exp(\g) \subset G$ generates $G$, there must be some $X_1, X_2, \dots X_n, Y_1, Y_2, \dots, Y_m \in \g$ such that $\prod_{i=1}^n \exp(X_i)^{k_i} = A$ and $\prod_{i=1}^m \exp(Y_i)^{j_i} = B$ with $k_1, \dots, k_n, j_1, \dots, j_n \in \N$. 

        Since $\g$ is abelian, we have 
        \begin{align*}
            \prod_{i=1}^n \exp(X_i)^{k_i} &= \exp\left(\sum_{i=1}^n k_iX_i\right) = A\\ 
            \prod_{i=1}^n \exp(Y_i)^{j_i} &= \exp\left(\sum_{i=1}^m j_iY_i\right) = B
        \end{align*} 
        
        Since $\g$ is closed under addition and scalar multiplication, we may make the substitutions, 
        \begin{align*}
            X' = \sum_{i=1}^n k_iX_i\\ 
            Y' = \sum_{i=1}^m j_iY_i
        \end{align*}
        So 
        \[AB = \exp(X')\exp(Y')\] 
        and since $\g$ is abelian, $[X', Y'] = 0 \implies \exp(X')\exp(Y') = \exp(X' + Y')$ so 
        \[AB = \exp(X' + Y') = \exp(Y')\exp(X') = BA\]

        Therefore, $G$ is abelian. $\qed$
    \color{black}

\pagebreak


\section*{Problem 4}
Prove that $\exp: \mathfrak{g} \longrightarrow G$ is surjective if $\g$ is abelian.  (Note these exercises effectively prove that if $G$ is connected, compact, and abelian, then it is an $n$-torus)

    \color{blue}
        Exactly as in Problem 3, let $\g$ be the abelian lie algebra of a connected lie group $G$. Let $A \in G$. 

        Since $\exp(\g) \subset G$ generates $G$, there must be some $X_1, X_2, \dots X_n \in \g$ such that 
        \[\prod_{i=1}^n \exp(X_i)^{k_i} = A\]

        Since $\g$ is abelian, we have 
        \[\prod_{i=1}^n \exp(X_i)^{k_i} = \exp\left(\sum_{i=1}^n k_i X_i\right) = A\]
        
        Since this argument holds for all $A \in G$, every element of $G$ is the image of some element in $\g$ under the exponential map. Hence, $\exp$ is surjective if $\g$ is abelian and $G$ is connected. $\qed$
        
    \color{black}


\pagebreak

\section*{Problem 5}
Let $G$ be the lie-group
\[G = \left\{\begin{pmatrix}
    1 & a & b & c \\
    0 & 1 & 0 & d \\
    0 & 0 & 1 & e \\
    0 & 0 & 0 & 1
\end{pmatrix} \, \bigg\vert \, a,b,c,d,e \in \R\right\}\]
Calculate the lie-algebra $\g$.  Prove that the exponential map $\exp : \g \longrightarrow G$ is a bijection. Is it a group-homomorphism?

    \color{blue}
        \[\g = \R A \oplus \R B \oplus \R C \oplus \R D \oplus \R E\]
        where 
        \[A = \begin{pmatrix}
            0 & 1 & 0 & 0 \\
            0 & 0 & 0 & 0 \\
            0 & 0 & 0 & 0 \\
            0 & 0 & 0 & 0
        \end{pmatrix}\; B = \begin{pmatrix}
            0 & 0 & 1 & 0 \\
            0 & 0 & 0 & 0 \\
            0 & 0 & 0 & 0 \\
            0 & 0 & 0 & 0
        \end{pmatrix}\; C = \begin{pmatrix}
            0 & 0 & 0 & 1 \\
            0 & 0 & 0 & 0 \\
            0 & 0 & 0 & 0 \\
            0 & 0 & 0 & 0
        \end{pmatrix}\; D = \begin{pmatrix}
            0 & 0 & 0 & 0 \\
            0 & 0 & 0 & 1 \\
            0 & 0 & 0 & 0 \\
            0 & 0 & 0 & 0
        \end{pmatrix}\; E = \begin{pmatrix}
            0 & 0 & 0 & 0 \\
            0 & 0 & 0 & 0 \\
            0 & 0 & 0 & 1 \\
            0 & 0 & 0 & 0
        \end{pmatrix}\]
        then calculating brackets, we get 
        \begin{align*}
            [A, B] &= 0\\
            [A, C] &= 0\\ 
            [A, D] &= C\\ 
            [A, E] &= 0\\
            [B, C] &= 0\\ 
            [B, D] &= 0\\
            [B, E] &= C\\
            [C, D] &= 0\\
            [C, E] &= 0\\
            [D, E] &= 0\\
        \end{align*}

        Let $X \in \g$. Then
        \begin{align*}
            X &= \begin{pmatrix}
                0 & a & b & c\\
                0 & 0 & 0 & d\\
                0 & 0 & 0 & e\\
                0 & 0 & 0 & 0
            \end{pmatrix}\\ 
            X^2 &= \begin{pmatrix}
                0 & 0 & 0 & ad + be\\
                0 & 0 & 0 & 0\\
                0 & 0 & 0 & 0\\
                0 & 0 & 0 & 0
            \end{pmatrix}\\ 
            X^3 &= \begin{pmatrix}
                0 & 0 & 0 & 0\\
                0 & 0 & 0 & 0\\
                0 & 0 & 0 & 0\\
                0 & 0 & 0 & 0
            \end{pmatrix}
        \end{align*}
        so 
        \[\exp(X) = I + X + \frac{1}{2}X^2 = \begin{pmatrix}
            1 & a & b & c + \frac{ad + be}{2}\\ 
            0 & 1 & 0 & d\\
            0 & 0 & 1 & e\\
            0 & 0 & 0 & 1
        \end{pmatrix}\]

        Trivially, the map 
        \[\exp: \begin{pmatrix}
            0 & a & b & c\\
            0 & 0 & 0 & d\\
            0 & 0 & 0 & e\\
            0 & 0 & 0 & 0
        \end{pmatrix} \mapsto \begin{pmatrix}
            1 & a & b & c + \frac{ad + be}{2}\\ 
            0 & 1 & 0 & d\\
            0 & 0 & 1 & e\\
            0 & 0 & 0 & 1
        \end{pmatrix}\] 
        is a bijection. 

        However, it is not a group homomorphism: Let $Y  =\begin{pmatrix}
            0 & x & y & z\\
            0 & 0 & 0 & v\\
            0 & 0 & 0 & w\\
            0 & 0 & 0 & 0
        \end{pmatrix}$. Then
        \begin{align*}
            \exp(X)\exp(Y) &= \begin{pmatrix}
                1 & a & b & c + \frac{ad + be}{2}\\ 
                0 & 1 & 0 & d\\
                0 & 0 & 1 & e\\
                0 & 0 & 0 & 1
            \end{pmatrix}\begin{pmatrix}
                1 & x & y & cz+ \frac{xw + yv}{2}\\
                0 & 1 & 0 & v\\
                0 & 0 & 1 & w\\
                0 & 0 & 0 & 1
            \end{pmatrix}\\ 
            &= \begin{pmatrix}
                1 & a + x & b & c + z + \frac{ad + be}{2} + av + bw + \frac{vx + wy}{2}\\ 
                0 & 1 & 0 & d + w\\
                0 & 0 & 1 & e + v\\
                0 & 0 & 0 & 1
            \end{pmatrix}\\ 
            \exp(X + Y) &= \begin{pmatrix}
                1 & a + x & b + y & c + z + \frac{ad + be}{2} + \frac{av + bw}{2}+ \frac{dx + ey}{2} + \frac{vx + wy}{2}\\ 
                0 & 1 & 0 & d + v\\
                0 & 0 & 1 & e + w\\
                0 & 0 & 0 & 1 
            \end{pmatrix}
        \end{align*}
    
    \color{black}

\pagebreak
\section*{Problem 6}
Write $G$ in Problem 5 as a non-split central extension of $\R^{4}$ by $\R$ and a split non-central extension of $\R^{2}$ by $\R^{3}$.  That is, show there exists short exact sequences
\[
\R \xrightarrow{i_{1}} G \xrightarrow{p_{1}} \R^{4} \text{ and }\R^{3} \xrightarrow{i_{2}}  G  \xrightarrow{p_{2}}  \R^{2}
\]
where the map $p_{1} : G \longrightarrow \R^{4}$ admits no section and the map $i_{1}: \R \longrightarrow G$ takes its image in the center of $G$, and, for the second sequence, the map $p_{2}: G \longrightarrow \R^{2}$ admits a section, and the map $i_{2}: \R^{3} \longrightarrow G$ takes it image outside the center of $G$.  

    \color{blue}    
        From problem 5, we have 
        \[G = \begin{pmatrix}
            1 & a & b & c \\
            0 & 1 & 0 & d \\
            0 & 0 & 1 & e \\
            0 & 0 & 0 & 1 
        \end{pmatrix}\]

        First, we will construct the sequence 
        \[\R \hookrightarrow G \twoheadrightarrow \R^4\]

        The center of $G$ is the set of matrices of the form
        \[Z(G) = \begin{pmatrix}
            1 & 0 & 0 & x\\
            0 & 1 & 0 & 0\\
            0 & 0 & 1 & 0\\
            0 & 0 & 0 & 1
        \end{pmatrix}\]

        Thus for $\im i = Z(G)$, we can take $i: \R \to G$ to be
        \[i(x) = \begin{pmatrix}
            1 & 0 & 0 & x\\
            0 & 1 & 0 & 0\\
            0 & 0 & 1 & 0\\
            0 & 0 & 0 & 1
        \end{pmatrix}\]
        and clearly, 
        \[i(x + y) = \begin{pmatrix}
            1 & 0 & 0 & x + y\\
            0 & 1 & 0 & 0\\
            0 & 0 & 1 & 0\\
            0 & 0 & 0 & 1
        \end{pmatrix} = \begin{pmatrix}
            1 & 0 & 0 & x\\
            0 & 1 & 0 & 0\\
            0 & 0 & 1 & 0\\
            0 & 0 & 0 & 1
        \end{pmatrix}\begin{pmatrix}
            1 & 0 & 0 & y\\
            0 & 1 & 0 & 0\\
            0 & 0 & 1 & 0\\
            0 & 0 & 0 & 1
        \end{pmatrix} = i(x)i(y)\]

        Then for $\ker p = \im i$, the obvious choice for $p: G \to \R^4$ is
        \[p\begin{pmatrix}
            1 & a & b & c\\
            0 & 1 & 0 & d\\
            0 & 0 & 1 & e\\
            0 & 0 & 0 & 1 
        \end{pmatrix} = (a, b, d, e)\]
        and again, 
        \begin{align*}
            p\begin{pmatrix}
                1 & a & b & c\\
                0 & 1 & 0 & d\\
                0 & 0 & 1 & e\\
                0 & 0 & 0 & 1 
            \end{pmatrix} p\begin{pmatrix}
                1 & x & y & z\\ 
                0 & 1 & 0 & r\\ 
                0 & 0 & 1 & s\\
                0 & 0 & 0 & 1
            \end{pmatrix} &= (a, b, d, e) + (x, y, r, s) = (a + x, b + y, d +r, e + s)\\ 
            p\left(\begin{pmatrix}
                1 & a & b & c\\
                0 & 1 & 0 & d\\
                0 & 0 & 1 & e\\
                0 & 0 & 0 & 1 
            \end{pmatrix} \begin{pmatrix}
                1 & x & y & z\\ 
                0 & 1 & 0 & r\\ 
                0 & 0 & 1 & s\\
                0 & 0 & 0 & 1
            \end{pmatrix}\right) &= p\begin{pmatrix}
                1 & a + x & b + y & c + z + ar + bs\\ 
                0 & 1 & 0 & d + r \\ 
                0 & 0 & 1 & e + s\\ 
                0 & 0 & 0 & 1
            \end{pmatrix} = (a + x, b + y, d +r, e + s)
        \end{align*}
        so we do have a short exact sequence. However, $p$ admits no section: Let 
        \[\sigma(a, b, d, e) = \begin{pmatrix}
            1 & a & b & f(a, b, d, e)\\ 
            0 & 1 & 0 & d\\ 
            0 & 0 & 1 & e\\ 
            0 & 0 & 0 & 1
        \end{pmatrix}\]
        for some function $f: \R^4 \to \R$. Then, 
        \[p\sigma(a, b, d, e) = p\begin{pmatrix}
            1 & a & b & f(a, b, d, e)\\ 
            0 & 1 & 0 & d\\ 
            0 & 0 & 1 & e\\ 
            0 & 0 & 0 & 1
        \end{pmatrix} = (a, b, d, e) \quad \checkmark\] 
        But $\sigma$ is not a homomorphism:
        \begin{align*}
            \sigma(a + x, b + y, d + z, e + w) &= \begin{pmatrix}
                1 & a+x & b+y & f(*)\\ 
                0 & 1 & 0 & d+z\\ 
                0 & 0 & 1 & e+w\\ 
                0 & 0 & 0 & 1
            \end{pmatrix}\\ 
            \sigma(a, b, d, e)\sigma(x, y, z, w) &= \begin{pmatrix}
                1 & a & b & f(a, b, d, e)\\ 
                0 & 1 & 0 & d\\ 
                0 & 0 & 1 & e\\ 
                0 & 0 & 0 & 1
            \end{pmatrix} \begin{pmatrix}
                1 & x & y & f(x, y, z, w)\\ 
                0 & 1 & 0 & z\\ 
                0 & 0 & 1 & w\\ 
                0 & 0 & 0 & 1
            \end{pmatrix}\\ 
            &= \begin{pmatrix}
                1 & a_x & b+y & f(a, b, d, e) + f(x, y, z, w) + bw + az\\ 
                0 & 1 & 0 & d+z\\ 
                0 & 0 & 1 & e+w\\ 
                0 & 0 & 0 & 1
            \end{pmatrix}\\ 
            \sigma(a + x, b + y, d + z, e + w) &\neq \sigma(a, b, d, e)\sigma(x, y, z, w)
        \end{align*}

        \vspace*{10pt}
        \hrule 
        \vspace*{10pt}

        Now, we will construct the sequence
        \[\R^3 \hookrightarrow G \twoheadrightarrow \R^2\]

        We want $i: \R^3 \to G$ to take its image outside the center of $G$. Let 
        \[i(x, y, z) = \begin{pmatrix}
            1 & x & 0 & y\\ 
            0 & 1 & 0 & 0\\
            0 & 0 & 1 & z\\
            0 & 0 & 0 & 1
        \end{pmatrix} \notin Z(G)\]

        Then let
        \[p\begin{pmatrix}
            1 & a & b & c\\ 
            0 & 1 & 0 & d\\
            0 & 0 & 1 & e\\
            0 & 0 & 0 & 1
        \end{pmatrix} = (b, d)\]
        so 
        \[\ker p = \begin{pmatrix}
            1 & a & 0 & c\\ 
            0 & 1 & 0 & 0\\
            0 & 0 & 1 & e\\
            0 & 0 & 0 & 1
        \end{pmatrix} = \im i\] 

        Then both $i$ and $p$ are homomorphisms:
        \begin{align*}
            i(x+a, y+b, z+c) &= \begin{pmatrix}
                1 & x+a & 0 & y+b\\ 
                0 & 1 & 0 & 0\\
                0 & 0 & 1 & z+c\\
                0 & 0 & 0 & 1
            \end{pmatrix} = \begin{pmatrix}
                1 & x & 0 & y\\ 
                0 & 1 & 0 & 0\\
                0 & 0 & 1 & z\\
                0 & 0 & 0 & 1
            \end{pmatrix} \begin{pmatrix}
            1 & a & 0 & b\\ 
            0 & 1 & 0 & 0\\
            0 & 0 & 1 & c\\
            0 & 0 & 0 & 1
        \end{pmatrix} = i(x, y,z)i(a, b, c)
    \end{align*}
    \begin{align*}
        p\left(\begin{pmatrix}
            1 & a & b & c\\ 
            0 & 1 & 0 & d\\
            0 & 0 & 1 & e\\
            0 & 0 & 0 & 1
        \end{pmatrix}\begin{pmatrix}
            1 & x & y & z\\ 
            0 & 1 & 0 & q\\
            0 & 0 & 1 & r\\
            0 & 0 & 0 & 1
        \end{pmatrix}\right) &= p\begin{pmatrix}
            1 & a + x & b + y & *\\ 
            0 & 1 & 0 & d + q\\ 
            0 & 0 & 1 & e + r\\ 
            0 & 0 & 0 & 1
        \end{pmatrix} = (a+x, e+r)\\ 
        p\begin{pmatrix}
            1 & a & b & c\\ 
            0 & 1 & 0 & d\\
            0 & 0 & 1 & e\\
            0 & 0 & 0 & 1
        \end{pmatrix} + p\begin{pmatrix}
            1 & x & y & z\\ 
            0 & 1 & 0 & q\\
            0 & 0 & 1 & r\\
            0 & 0 & 0 & 1
        \end{pmatrix} &= (a, e) + (x, r) = (a + x, e + r)
        \end{align*}

        Finally, $p$ admits a section. Let 
        \[\sigma(a, b) = \begin{pmatrix}
            1 & a & 0 & a+b\\ 
            0 & 1 & 0 & 0\\ 
            0 & 0 & 0 & b\\ 
            0 & 0 & 0 & 1
        \end{pmatrix}\]

        Then 
        \[p\sigma(a, b) = p\begin{pmatrix}
            1 & a & 0 & a+b\\ 
            0 & 1 & 0 & 0\\ 
            0 & 0 & 0 & b\\ 
            0 & 0 & 0 & 1
        \end{pmatrix} = (a, b) \quad \checkmark\]
        and $\sigma$ is a homomorphism: 
        \begin{align*}
            \sigma(a + x, b+y) &= \begin{pmatrix}
                1 & a+x & 0 & a+x+b+y\\ 
                0 & 1 & 0 & 0\\ 
                0 & 0 & 0 & b+y\\ 
                0 & 0 & 0 & 1
            \end{pmatrix}\\ 
            \sigma(a, b)\sigma(x, y) &= \begin{pmatrix}
                1 & a & 0 & a+b\\ 
                0 & 1 & 0 & 0\\ 
                0 & 0 & 0 & b\\ 
                0 & 0 & 0 & 1
            \end{pmatrix}\begin{pmatrix}
                1 & x & 0 & x+y\\ 
                0 & 1 & 0 & 0\\ 
                0 & 0 & 0 & y\\ 
                0 & 0 & 0 & 1
            \end{pmatrix} = \begin{pmatrix}
                1 & a+x & 0 & a+x+b+y\\ 
                0 & 1 & 0 & 0\\ 
                0 & 0 & 0 & b+y\\ 
                0 & 0 & 0 & 1
            \end{pmatrix}
        \end{align*}
    \color{black}

\pagebreak
\section*{Problem 7}
Let $U_{3} \subset \GL(3,\R)$ denote the invertible upper triangular matrices.  Prove that $H$, the Heisenberg group, sits normally inside $U_{3}$ and has an abelian quotient.  Is this short exact sequence split?

    \color{blue}
        $H$ is the group of matrices of the form 
        \[A = \begin{pmatrix}
            1 & c & a\\
            0 & 1 & b\\
            0 & 0 & 1
        \end{pmatrix}\]

        Let $B = \begin{pmatrix}
            x & y & z\\ 
            0 & w & v\\ 
            0 & 0 & u
        \end{pmatrix} \in U_3$. Then 
        \[BAB^{-1} = \begin{pmatrix}
            x & y & z\\ 
            0 & w & v\\ 
            0 & 0 & u
        \end{pmatrix}\begin{pmatrix}
            1 & c & a\\
            0 & 1 & b\\
            0 & 0 & 1
        \end{pmatrix}\begin{pmatrix}
            \frac{1}{x} & -\frac{y}{xw} & \frac{yv - wz}{xwu}\\
            0 & \frac{1}{w} & -\frac{v}{wu}\\
            0 & 0 & \frac{1}{u}
        \end{pmatrix} = \begin{pmatrix}
            1 & \frac{cx}{w} & \frac{awx - cvx + bwy}{uw}\\ 
            0 & 1 & \frac{bw}{u}\\
            0 & 0 & 1
        \end{pmatrix} \in H\]
        Therefore, $H \trianglelefteq U_3$. 

        \vspace*{10pt}
        \hrule 
        \vspace*{10pt}

        $U_3/H$ is abelian if for any $g \in U_3$ and $h \in H$, 
        \[ghg^{-1}h^{-1} \in H\]

        Using $A$ and $B$ above (and the normality calculation) we may calculate this product explicitly:
        \begin{align*} 
            BAB^{-1}A^{-1} &= \begin{pmatrix}
                1 & \frac{cx}{w} & \frac{awx - cvx + bwy}{uw}\\ 
                0 & 1 & \frac{bw}{u}\\
                0 & 0 & 1
            \end{pmatrix} \begin{pmatrix}
                1 & -c & bc - a\\ 
                0 & 1 & -b\\
                0 & 0 & 1
            \end{pmatrix}\\ 
            &= \begin{pmatrix}
                1 & -\frac{cw - cx}{w} & -\frac{auw - awx + cvx - bwy - bcuw + bcux}{uw}\\
                0 & 1 & -\frac{b(u - w)}{u}\\
                0 & 0 & 1
            \end{pmatrix}
        \end{align*} 
        Clearly, this is in $H$ so the quotient is abelian.

        \vspace*{10pt}
        \hrule 
        \vspace*{10pt}

        This gives us the short exact sequence 
        \[H \overset{i}{\hookrightarrow} U_3 \overset{p}{\twoheadrightarrow} U_3/H\]
        
        Then we need $\ker p = \im i$. Since the natural $i$ is simple inclusion, we need a homomorphism $p: U_3 \to U_3/H$ whose kernel is 
        \[\begin{pmatrix}
            1 & c & a\\
            0 & 1 & b\\
            0 & 0 & 1
        \end{pmatrix}\]

        Let $p: U_3 \to \R^3$ by 
        \[p\begin{pmatrix}
            x & y & z\\ 
            0 & w & v\\ 
            0 & 0 & u
        \end{pmatrix} \mapsto (x, w, u)\]

        Then $\ker p = H$ and the map is surjective and a homomorphism:
        \begin{align*}
            p\left(\begin{pmatrix}
                x_1 & y_1 & z_1\\ 
                0 & w_1 & v_2\\ 
                0 & 0 & u_2
            \end{pmatrix}\begin{pmatrix}
                x_2 & y_2 & z_2\\ 
                0 & w_2 & v_2\\ 
                0 & 0 & u_2
            \end{pmatrix}\right) &= p \begin{pmatrix}
                x_2x_1 & * & *\\ 
                0 & w_1w_2 & *\\
                0 & 0 & u_2u_1
            \end{pmatrix} = (x_2x_1, w_1w_2, u_2u_1)\\
            p\begin{pmatrix}
                x_1 & y_1 & z_1\\ 
                0 & w_1 & v_2\\ 
                0 & 0 & u_2
            \end{pmatrix}p\begin{pmatrix}
                x_2 & y_2 & z_2\\ 
                0 & w_2 & v_2\\ 
                0 & 0 & u_2 
            \end{pmatrix} &= (x_1, w_1, u_1)(x_2, w_2, u_2) = (x_1x_2, w_1w_2, u_1u_2)
        \end{align*}
        so $U_3/\ker p = U_3/H \simeq \R^3$. 

        So $p: U_3 \to U_3/H$ and the sequence is short exact as desired. To show it is split, we need to find a homomorphism $\sigma: U_3/H \to U_3$ such that $p\sigma = \text{1}_{\R^3}$.

        Let 
        \[\sigma(x, y, z) = \begin{pmatrix}
            x & 0 & 0\\
            0 & y & 0\\
            0 & 0 & z
        \end{pmatrix}\]
        so 
        \[p\sigma(x, y, z) = p\begin{pmatrix}
            x & 0 & 0\\
            0 & y & 0\\
            0 & 0 & z
        \end{pmatrix} = (x, y, z)\]

        Further, this is a homomorphism: 
        \begin{align*}
            \sigma(xa, yb, zc) &= \begin{pmatrix}
                xa & 0 & 0\\
                0 & yb & 0\\
                0 & 0 & zc
            \end{pmatrix}\\ 
            \sigma(x, y, z)\sigma(a, b, c) &= \begin{pmatrix}
                x & 0 & 0\\
                0 & y & 0\\
                0 & 0 & z
            \end{pmatrix}\begin{pmatrix}
                a & 0 & 0\\
                0 & b & 0\\
                0 & 0 & c
            \end{pmatrix} = \begin{pmatrix}
                xa & 0 & 0\\
                0 & yb & 0\\
                0 & 0 & zc
            \end{pmatrix}
        \end{align*}
        
        So we are done. $\qed$
    \color{black}

\end{document}