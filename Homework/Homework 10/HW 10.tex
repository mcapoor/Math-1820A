\documentclass[12pt]{article}

%%%%%%%%%%%%%%%%%%%%%%
%%%%%%% PACAKAGES %%%%%%%
%%%%%%%%%%%%%%%%%%%%%


%%%%%%%%%%%%% GRAPHICS/FONTS
\usepackage{amsmath,amsfonts,amssymb,graphicx,tikz-cd,pgfplots}

%%%%%%%%%%%%%% FORMATTING
\usepackage{geometry,titlesec,hyperref,xhfill,setspace,float,fancyhdr}
\usepackage{parskip}

%%%%%%%%DEFINING COMMANDS
\usepackage{xifthen}

%%%%%%%%%TIKZ STUFF
\usetikzlibrary{positioning,calc}
\usetikzlibrary{decorations.markings}
\usepgfplotslibrary{polar}
\usepgflibrary{shapes.geometric}
\usepgfplotslibrary{fillbetween}

%%%%%%FOR THE GRAPHS
\pgfplotsset{my style/.append style={axis x line=middle, axis y line=
middle, xlabel={$x$}, ylabel={$y$}, axis equal }}

%%%%%%MARGINS
\geometry{
    letterpaper,
    left=0.25in,
    right=0.25in,
    top=0.25in,
    bottom=0.25in
}

%%%%%%MAKES THE HEADER AND FOOTER FANCY
\fancyhf{}
\renewcommand{\headrulewidth}{0pt}
\pagestyle{fancy}

%%%%%%%%%%%%%%%%%%%%%%
%%%%%%% COMMANDS %%%%%%%
%%%%%%%%%%%%%%%%%%%%%
\newcommand{\underscore}{\underline{\hspace{2mm}}}
\newcommand{\Z}{\mathbb{Z}}
\newcommand{\N}{\mathbb{N}}
\newcommand{\Q}{\mathbb{Q}}
\newcommand{\C}{\mathbb{C}}
\newcommand{\R}{\mathbb{R}}
\newcommand{\Aut}{\text{Aut}}
\newcommand{\GL}{\text{GL}}
\newcommand{\SL}{\text{SL}}
\newcommand{\SO}{\text{SO}}
\newcommand{\PGL}{\text{PGL}}
\newcommand{\Stab}{\text{Stab}}
\newcommand{\End}{\text{End}}
\newcommand{\ad}{\text{ad}}
\newcommand{\lra}{\longrightarrow}

\newcommand{\gl}{\mathfrak{gl}}
\newcommand{\sll}{\mathfrak{sl}}
\newcommand{\pgl}{\mathfrak{pgl}}
\newcommand{\g}{\mathfrak{g}}
\newcommand{\h}{\mathfrak{h}}
\newcommand{\n}{\mathfrak{n}}
\newcommand{\la}{\mathfrak{a}}
\newcommand{\lb}{\mathfrak{b}}


\newcommand{\RP}{\mathbb{R}P}
\newcommand{\K}{\emph{K}}

\newcommand{\Ha}{\mathbb{H}}

\newcommand{\norm}[1]{\left\vert \left\vert #1 \right\vert \right\vert}

\newcommand{\qed}{\quad \blacksquare}

%TEXT COMMAND
\newcommand{\T}[1][]{\text{#1}}
\newcommand{\TB}[1][]{\mathbb{#1}}

\newcommand{\ihat}{\hat \imath}
\newcommand{\jhat}{\hat \jmath}
\newcommand{\khat}{\hat k}

\newcommand{\xlra}[1][]{%
  \ifthenelse{\isempty{#1}}%
    {\xrightarrow{\phantom{,,,,,,}}}% if #1 is empty
    {\xrightarrow{\phantom{,,}#1\phantom{,,}}}% if #1 is not empty
}


%%%%%%%%%%%%%%%%%%%%%%%%%%%%%%%%%%%%%%%%
%%%%%%%%BEGINNING OF ACTUAL DOCUMENT %%%%%%%%%%
%%%%%%%%%%%%%%%%%%%%%%%%%%%%%%%%%%%%%%%%



%%%%TITLE______REMEMBER TO REGULARLY CHANGE THIS!!!!
\title{Math 1820A Spring 2024 - Homework 10}
\date{}

\begin{document}
\maketitle
\vspace{-0.5in}
%%%%%%%%%%%%%%%%
\begin{spacing}{1.5}
\noindent \textbf{Instructions:}  This assignment is worth twenty points.  Please complete the following problems assigned below.  Submissions with insufficient explanation may lose points due to a lack of reasoning or clarity.  If you are handwriting your work, please ensure it is readable and well-formatted for the grader.

Be sure when uploading your work to \textbf{assign problems to pages}.  Problems with pages not assigned to them \textbf{may not be graded}.  
\end{spacing}

\noindent
\textbf{Additional Problems:}   For these problems let $\Ha$ denote the algebra of Hamiltonians and $\Ha^{\times}$ denote all the non-zero elements as a \emph{group} under multiplication.  For the context of this homework, assume a `rotation' is orientation preserving, and a `reflection' is preserves a codimension 1 subspace.  By a circle on an $n$-sphere we mean the intersection of an affine plane with $S^{n}$.  By a circle in $\R^{n}$ we mean a typical Euclidean circle, or, also a line. 

1.  Let $H : S^{3} \lra S^{2}$ be the Hopf-fibration map as in class, namely $H(q) = q\hat{k}q^{-1}$ where here we are identifying $S^{3} \subset \Ha$ with the 3-sphere of unit quaternions.  Prove that the fiber over a point $v \neq -\hat{k}$ in $S^{2}$ is given by
\[H^{-1}(v) = \frac{1}{\sqrt{2+2c}}\left(1 + c - b\hat{i} + a \hat{j}\right)S \]
where $v = (a,b,c) \in S^{2}$ and $S=\text{Stab}_{\hat{k}} = \left\{\cos(\theta) + \sin(\theta)\hat{k} \, | \, \theta \in S^{1} \right\}$.  (Note in class we technically proved that everything in here is in the fiber, but we didn't prove that was entire fiber)

    \color{blue}
        Together with what we proved in class, it suffices to show that this expression is the entire fibre, i.e. there does not exist $p \in S^3$ for which $H(p) = v = (a, b, c)$ but where 
        \[p \neq \frac{1}{\sqrt{2 + 2c}}\left(1 + c - b\ihat + a\jhat\right)\left(\cos \theta + \sin \theta \khat\right)\]
        for any $\theta \in S^1$. 

        Suppose such a $p$ exists. Then it must be the fibre of some $v \in S^2$, i.e. $p = H^{-1}(v)$. 

        However, if we have any point $p \in S^3$ which is in the fibre of $v$, then we can write the entire fibre as 
        \[H^{-1}(v) = p \text{Stab}(\khat) = p(\cos \theta + \sin \theta \khat)\] 
        because the Hopf fibration is a principal $S^1$-bundle over $S^2$ so for $s \in S^1$, 
        \[H(qs) = qs\khat (qs)^{-1} = qs\khat s^{-1}q^{-1} = q\khat q^{-1} = H(q)\]
        and for $q_1, q_2 \in S^3$, 
        \[H(q_1 q_2) = q_1 q_2 \khat q_2^{-1} q_1^{-1} = q_1H(q_2)q_1^{-1} = I_{q_1}(H(q_2))\]
        is a rotation of $H(q_2)$ about the vector part of $q_1$. 

        In class, we defined $q$ as the quaternion in $S^3$ which rotated $\khat$ to $v$ for all $v \neq -\khat$ so $H^{-1}(v) = q\khat S = qS$. 

        Since $p \neq q$, we have 
        \[pS = H^{-1}(v) = qS\]
        but this is a contradiction. $\qed$ 
    \color{black}


\pagebreak

2.  Let $U \subset S^{2}$ be defined by $U = S^{2}\setminus -\hat{k}$.  Construct a map from $\Phi:  H^{-1}(U) \lra U\times S^{1}$ via 
\[\Phi(q) = \left(v, \theta\right)\]
where $v = H(q) \in S^{2}$ and $\theta \in S^{1}$ is the unique $\theta \in S^{1}$ for which 
\[q = \frac{1}{\sqrt{2+2c}}\left(1 + c - b\hat{i} + a \hat{j}\right)\left(\cos(\theta) + \sin(\theta)\hat{k}\right)\]
Prove this map is bijective, thus $\Phi^{-1} : H^{-1}(U) \lra U\times S^{1}$ exists.

    \color{blue}
        We first show that $\Phi$ is injective.

        Let $v_1 = H(q_1)$ and $v_2 = H(q_2)$ associated to $\theta_1, \theta_2 \in S^1$ respectively for $q_1, q_2 \in S^3$. 

        Suppose $v_1 = v_2$. By definition, the Hopf fibration takes circles in $S^3$ to points in $S^2$. Thus, $q_1$ and $q_2$ are on the same circle in $S^3$, i.e. they are the same up to $\theta$. 
        
        If $\theta_1 = \theta_2$, then $q_1 = q_2$ and $\Phi(q_1) = \Phi(q_2)$.

        If $\theta_1 \neq \theta_2$ then $(v_1, \theta_1) \neq (v_2, \theta_2)$ and $\Phi(q_1) \neq \Phi(q_2)$ by definition. Therefore, $\Phi(q_1) = \Phi(q_2)$ iff $q_1 = q_2$ and $\theta_1 = \theta_2$. Hence, it is injective. 

        Now we want to show surjectivity. Since it is a fibre bundle, $H$ is surjective. Further, we are certainly free to choose any $\theta \in [0, 2\pi)$ since for a given $v \in U$, all the points in the fibre are the same up to $\theta$. Thus, for any $(v, \theta) \in U \times S^1$, we can find a $q \in S^3$ such that $H(q) = v$ and $q$ is associated to a unique $\theta$. 

        Since $\Phi$ is surjective and injective, it is bijective. $\qed$           
    \color{black}

\pagebreak

3.  Show that $\Phi$ as defined in Problem 2 satisfies the following commutative diagram.
\[\begin{tikzcd}
H^{-1}(U) \arrow{r}{\Phi} \arrow{dr}[swap]{H} & U\times S^{1} \arrow{d}{p_{1}} \\
& U
\end{tikzcd}\]
where $p_{1}: U \times S^{1} \lra U$ is projection onto the first factor.  Moreover, explain why both $\Phi$ and $\Phi^{-1}$ as defined in Problem 2 are both continuous. 

    \color{blue}
        If we define $p_1: U \times S^1 \to U$ by $p_1(v, \theta) = v$, then 
        \[p_1 \circ \Phi(q) = p_1(v, \theta) = v\]
        for all $q \in H^{-1}(U)$. 

        Let $v \in U \subset S^2$. Then 
        \[H(H^{-1}(v)) = H(q) = v\]

        Thus, the diagram commutes.

        \vspace*{10pt}
        \hrule 
        \vspace*{10pt}

        Since the fibres of the Hopf map are circles in $S^3$ parameterized by $\theta$, we can see that $\Phi$ is continuous.

        For $\Phi^{-1}$ we can see 
        \[\Phi^{-1}(v, \theta) = \frac{1}{\sqrt{2 + 2c}}(1 + c - b\ihat + a\jhat)(\cos \theta + \khat \sin \theta)\]
        has a discontinuity only at $c \geq -1$. Since $v = (a, b, c) \in U$, 
        \[c = -1 \implies v = (0, 0, -1)\]
        but $U = S^2\setminus -\khat$ so this is out of the domain and thus $\Phi^{-1}$ is continuous for all points in $U \times S^1$. $\qed$

    \color{black}


\pagebreak 

4.  Let $V \subset S^{2}$ be defined by $V= S^{2}\setminus \hat{k}$.  Construct a function $\Psi: H^{-1}(V) \lra V\times S^{1} $ that enjoys the same properties that $\Phi$ does, namely construct a $\Psi :  H^{-1}(V) \lra V\times S^{1}$ that is a continuous bijection with a continuous inverse which satisfies the diagram below
\[\begin{tikzcd}
H^{-1}(V) \arrow{r}{\Psi} \arrow{dr}[swap]{H} & V\times S^{1} \arrow{d}{p_{1}} \\
& V
\end{tikzcd}\]
    \color{blue}
        Consider $\Psi(q) = (v, \theta)$ where $v = H(q) \in V$ and $\theta \in S^1$ is the unique $\theta$ for which
        \[q = \sqrt{\frac{1 - c}{2}}\left(1 - \frac{a}{1 - c}\jhat + \frac{b}{1 - c}\khat\right)(\cos \theta + \sin \theta \khat)\]

        Clearly, $\Psi$ is a map $H^{-1}(V) \to V \times S^1$ since $q \in S^3$: 
        \begin{align*}
            \norm{q} &= \norm{\sqrt{\frac{1 - c}{2}}\left(1 - \frac{a}{1 - c}\jhat + \frac{b}{1 - c}\khat\right)(\cos \theta + \sin \theta \khat)}\\ 
            &= \norm{\sqrt{\frac{1 - c}{2}}\left(\cos \theta + \sin \theta \khat - \frac{a}{1 - c}\cos \theta \jhat - \frac{a}{1 - c} \sin \theta \khat\jhat + \frac{b}{1 - c} \cos \theta \khat + \frac{b}{1 - c} \sin \theta \khat \khat\right)}\\ 
            &= \norm{\sqrt{\frac{1- c}{2}}\left((\cos \theta - \frac{b}{1 - c}\sin \theta) + \frac{a}{1 - c}\sin \theta \ihat - \frac{a}{1- c}\cos \theta \khat + (\sin \theta + \frac{b}{1 -c} \cos \theta)\khat\right)}\\ 
            &= \sqrt{-\frac{a^2 + b^2 + c^2 - 2c + 1}{2c - 2}}\\ 
            &= \sqrt{-\frac{-2c + 2}{2c - 2}} = 1
        \end{align*}
        for all $(a, b, c) \in V = S^2\setminus \khat$

        Further, for $v \in V$, $H(H^{-1}(v)) = H(q) = v$ and 
        \[p_1(\Psi(H^{-1}(v))) = p_1(\Psi(q)) = p_1(v, \theta) = v\]
        so the diagram commutes. 

        It is bijective and continuous for the same reasons as $\Phi$. 
        
        Its inverse $\Psi^{-1}: V \times S^1 \to H^{-1}(V)$ is given by 
        \[\Psi^{-1}(v, \theta) = \sqrt{\frac{1 - c}{2}}\left(1 - \frac{a}{1 - c}\jhat + \frac{b}{1 - c}\khat\right)(\cos \theta + \sin \theta \khat)\]
        which has discontinuities only at $v = (0, 0, 1)$ which is not in $V$.
    \color{black}

\pagebreak 
5.  Let $w \in U \cap V$.  Consider the composition $(U\cap V)\times S^{1} \lra (U\cap V)\times S^{1}$ given by $\Psi \circ \Phi^{-1} : (U\cap V)\times S^{1} \lra (U\cap V)\times S^{1}$.  Prove that the composition $\Psi\circ \Phi^{-1}$ is of the form,
\[(\Psi \circ \Phi^{-1})(w,\theta) = \left(w, g_{v}(\theta)\right)\]
where $g_{w}: S^{1} \lra S^{1}$ is a symmetry of $S^{1}$ for each choice of $w \in U \cap V$.  Explicitly calculate this $g_{w}$ for each choice of $w \in U \cap V$.  

    \color{blue}
        Suppose $\theta \in S^1$ is the unique angle corresponding to each $w \in U \cap V = S^2\setminus \{\khat, -\khat\}$. 

        So 
        \[(\Psi \circ \Phi^{-1})(w, \theta) = \Psi(\Phi^{-1}(w, \theta)) = \Psi(H^{-1}(w)) = (H(H^{-1}(w)), \phi) = (w, \phi)\]
        where $\phi = g_w(\theta)$ is some symmetry of $S^1$ because the quaternion is determined only up to phase in the fibre. 

        Explicitly, we want to find the rotation between $q = \Phi^{-1}(w, \theta)$ and $p = \Psi^{-1}(w, \phi)$. Since they are in the same fibre, they are rotations within a single circle. 

        We know 
        \[qwq^{-1} = pwp^{-1} \implies p^{-1}qwq^{-1}p = (p^{-1}q)w(p^{-1}q)^{-1} = w\]
        Thus $(p^{-1}q)$ is a rotation fixing $w$, i.e. 
        \[p^{-1}q = \cos \frac{\theta}{2} + w\sin \frac{\theta}{2}\]
        
        But indeed, this is precisely what we want! This $\theta$ is the angle between $p$ and $q$ clockwise in the plane of the fibre. 

        Thus, for $w = (a, b, c) \in U \cap V$, 
        \[g_w(\theta) = (2\text{acos}(\theta), 0, 0, 0) \cdot p^{-1}q \]
        where 
        \begin{align*}
            p &= \sqrt{\frac{1 - c}{2}}\left(1 - \frac{a}{1 - c}\jhat + \frac{b}{1 - c}\khat\right)\\
            q &= \frac{1}{\sqrt{2 + 2c}}(1 + c - b\ihat + a\jhat)
        \end{align*}
    \color{black}


\pagebreak 
6.  Let $C \subset S^{2} \subset \Ha$ be defined as the set of all points in $\Ha$ of unit norm spanned by $\{\hat{i}, \hat{j}\}$.  Consider $H^{-1}(C) \subset S^{3}$ under stereographic projection $F: S^{3}\setminus \hat{k} \lra \R^{3}$.  Prove the image of $H^{-1}(C)$ under $F$ is a torus in $\R^{3}$. 

    \color{blue}
        By definition, 
        \[C = \{q \in \Ha \; | \; \norm{q} = 1, \; q = \R \ihat + \R \jhat\} = \{a \ihat + b\jhat \; | \; a, b \in \R,\; a^2 + b^2 = 1\}\]

        By the norm constraint, $a, b \in [0, 1]$ so we can write 
        \[C = \{\cos \theta \ihat + \sin \theta \jhat \; | \; \theta \in S^1\}\]
        which is clearly a circle in $S^2$.

        $H$ is a fibre bundle so 
        \[H^{-1}(C) = C \times S^1 = S^1 \times S^1 \simeq T^2 \subset S^3\]
    
        Since $H^{-1}(C)$ is a torus contained in $S^3$, it is equivalent to the Clifford Torus which stereographically projects onto the torus of rotation in $\R^3$ $\qed$

    \color{black}

\pagebreak 
7.  Using Problem 6, explain why $S^{3}$ is one solid torus glued to another. 

    \color{blue}
        Consider $S^3$ as $\R^3$ plus a point at infinity (as under the stereographic projection). 

        Let $T$ be a solid torus in $\R^3$ centered on the z-axis and consider $\R^3 \setminus T$. 

        We want to show that what remains is another solid torus. 

        Consider the hole left by $T$. Projected onto the plane, this hole is a disk. From problem 6, we know that the image of the fibre over a circle is a torus so the image of the fibre over a disk under stereographic projection will be a solid torus. 

        Locally, we can consider each the fibre of each point on the disk as a vertical line in $\R^3$ which is a circle in $S^3$. Since we want these to be disjoint, the circles must have larger and larger radius as we approach $(0, 0, 0)$ until at the origin, we have a circle of radius $\infty$ which is the point at infinity.

        Thus, all of $\R^3$ plus the point at infinity is covered by the two solid tori so $S^3$ is a solid torus glued to another. $\qed$

    \color{black}


\pagebreak  

8.  Prove that $S^{3}\setminus S^{1}$ is an $S^{1}$-bundle over $\R^{2}$.  Draw (to the best of your abilities) a cartoon of this; specifically if $g: S^{3}\setminus S^{1} \lra \R^{2}$ is your bundle map, what does $g^{-1}(U)$ look like for a small set $U$ about a point $v \in S^{2}$? 

    \color{blue}
        Let $g: S^3/S^1 \to \R^2$. Since $S^3 \setminus S^1 \simeq S^1 \times \R^2$, 
        \[g^{-1}: \R^2 \to \R^2 \times S^1\] 

        Therefore, a small set $U$ about a point $v \in S^2$ will be 
        \[g^{-1}(U) \simeq U \times S^1\]
        Hence, $S^3 \setminus S^1$ is an $S^1$-bundle over $\R^2$.

        \begin{center}
            \includegraphics*[width=0.8\textwidth]{S3 fibres.png}
        \end{center}

        Here, the bottom picture is the image of $\R^2$ with a small neighborhood $U$ around a point. This pullsback into a cylinder in $\R^2 \times S^1$. Since the fibre of each point is a circle, $g^{-1}(U)$ looks like a solid torus. 
    \color{black}


\pagebreak 

\textbf{Bonus: [3 pts]}.  Prove that $S^{2}$ is a not an $S^{1}$-bundle over $S^{1}$.

\textbf{Bonus: [3 pts]}.  Let $X$ be $S^{2}$.  Glue a copy of $D^{4}$ to $S^{2}$ via taking $\partial  D^{4} = S^{3}$ to $S^{2}$ through the Hopf-fibration.  Prove the resulting manifold is $\C P^{2}$, the complex projective plane. 

\textbf{Bonus: [2 pts]}.  Draw a cartoon illustrating that $S^{3}\setminus S^{1}$ is the same thing as $S^{1} \times \R^{2}$.  

\end{document}