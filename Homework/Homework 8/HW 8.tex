\documentclass[12pt]{article}

%%%%%%%%%%%%%%%%%%%%%%
%%%%%%% PACAKAGES %%%%%%%
%%%%%%%%%%%%%%%%%%%%%


%%%%%%%%%%%%% GRAPHICS/FONTS
\usepackage{amsmath,amsfonts,amssymb,graphicx,tikz-cd,pgfplots}

%%%%%%%%%%%%%% FORMATTING
\usepackage{geometry,titlesec,hyperref,xhfill,setspace,float,fancyhdr}

%%%%%%%%DEFINING COMMANDS
\usepackage{xifthen}

%%%%%%%%%TIKZ STUFF
\usetikzlibrary{positioning,calc}
\usetikzlibrary{decorations.markings}
\usepgfplotslibrary{polar}
\usepgflibrary{shapes.geometric}
\usepgfplotslibrary{fillbetween}

%%%%%%FOR THE GRAPHS
\pgfplotsset{my style/.append style={axis x line=middle, axis y line=
middle, xlabel={$x$}, ylabel={$y$}, axis equal }}

%%%%%%MARGINS
\geometry{
    letterpaper,
    left=0.25in,
    right=0.25in,
    top=0.25in,
    bottom=0.75in
}

\usepackage{parskip}
\setlength{\parindent}{0pt} 

%%%%%%MAKES THE HEADER AND FOOTER FANCY
\fancyhf{}
\renewcommand{\headrulewidth}{0pt}
\pagestyle{fancy}

%%%%%%%%%%%%%%%%%%%%%%
%%%%%%% COMMANDS %%%%%%%
%%%%%%%%%%%%%%%%%%%%%
\newcommand{\underscore}{\underline{\hspace{2mm}}}
\newcommand{\Z}{\mathbb{Z}}
\newcommand{\N}{\mathbb{N}}
\newcommand{\Q}{\mathbb{Q}}
\newcommand{\C}{\mathbb{C}}
\newcommand{\R}{\mathbb{R}}
\newcommand{\Aut}{\text{Aut}}
\newcommand{\GL}{\text{GL}}
\newcommand{\SL}{\text{SL}}
\newcommand{\SO}{\text{SO}}
\newcommand{\PGL}{\text{PGL}}
\newcommand{\Stab}{\text{Stab}}
\newcommand{\End}{\text{End}}
\newcommand{\ad}{\text{ad}}
\newcommand{\lra}{\longrightarrow}

\newcommand{\gl}{\mathfrak{gl}}
\newcommand{\sll}{\mathfrak{sl}}
\newcommand{\pgl}{\mathfrak{pgl}}
\newcommand{\g}{\mathfrak{g}}
\newcommand{\h}{\mathfrak{h}}
\newcommand{\n}{\mathfrak{n}}
\newcommand{\la}{\mathfrak{a}}
\newcommand{\lb}{\mathfrak{b}}


\newcommand{\RP}{\mathbb{R}P}
\newcommand{\K}{\emph{K}}

\newcommand{\Ha}{\mathbb{H}}
\newcommand{\brak}[1]{\left\langle #1 \right\rangle}
\newcommand{\abs}[1]{\left\vert #1 \right\vert}
%TEXT COMMAND
\newcommand{\T}[1][]{\text{#1}}
\newcommand{\TB}[1][]{\mathbb{#1}}

\newcommand{\norm}[1]{\left\vert \left\vert #1 \right\vert \right\vert}

\newcommand{\qed}{\quad \blacksquare}
\newcommand{\xlra}[1][]{%
  \ifthenelse{\isempty{#1}}%
    {\xrightarrow{\phantom{,,,,,,}}}% if #1 is empty
    {\xrightarrow{\phantom{,,}#1\phantom{,,}}}% if #1 is not empty
}


%%%%%%%%%%%%%%%%%%%%%%%%%%%%%%%%%%%%%%%%
%%%%%%%%BEGINNING OF ACTUAL DOCUMENT %%%%%%%%%%
%%%%%%%%%%%%%%%%%%%%%%%%%%%%%%%%%%%%%%%%



%%%%TITLE______REMEMBER TO REGULARLY CHANGE THIS!!!!
\title{Math 1820A Spring 2024 - Homework 8}
\date{}

\begin{document}
\maketitle
\vspace{-0.5in}
%%%%%%%%%%%%%%%%
\begin{spacing}{1.5}
\noindent \textbf{Instructions:}  This assignment is worth twenty points.  Please complete the following problems assigned below.  Submissions with insufficient explanation may lose points due to a lack of reasoning or clarity.  If you are handwriting your work, please ensure it is readable and well-formatted for the grader.

Be sure when uploading your work to \textbf{assign problems to pages}.  Problems with pages not assigned to them \textbf{may not be graded}.  
\end{spacing}




%%%%%%%%%%%%%%%%%%
\vspace{10mm}\noindent
\textbf{Textbook Problems: }  

\noindent
\textbf{Additional Problems:}   For these problems let $\Ha$ denote the algebra of Hamiltonians and $\Ha^{\times}$ denote all the non-zero elements as a \emph{group} under multiplication.  For the context of this homework, assume a `rotation' is orientation preserving, and a `reflection' is preserves a codimension 1 subspace. Also denote $L_{A}$ by a line and also the line-symmetry through $L_{A}$.

\pagebreak
1.  Let $L_{A}$ and $L_{B}$ be non-parallel lines through in $\R^{2}$ intersecting at $p$.  Prove that $L_{B}\circ L_{A}$ is a rotation in $\R^{2}$ about $p$ by an angle of $2\theta$ where $\theta$ is the oriented angle between $L_{A}$ and $L_{B}$.

    \color{blue}
        It suffices to construct a reference frame isomorphic to $\R^2$ with $p$ at the origin. Thus, $L_A$ and $L_B$ in the new frame are two line symmetries through the origin. By the Euler-Rodrigues Theorem (see Problem 7), $L_B \circ L_A$ is a rotation about $p$ by an angle of $2\theta$ in the clockwise direction where $\theta$ is the oriented angle between $L_A$ and $L_B$. $\qed$
    \color{black}


\pagebreak
2.  Let $L_{A}$ and $L_{B}$ be two parallel lines in $\R^{2}$.  Let $\Gamma = \langle L_{A}, L_{B} \rangle$.  Does there exist a point $P \in \R^{2}$ preserved by $\Gamma$, i.e. is there a point $P \in \R^{2}$ such that $\Gamma P = P$?  What about a line?

    \color{blue}
        Notice that $L_A \circ L_A = I$ and $L_B \circ L_B = I$ by the properties of reflection in $\R^2$. Further, we know the only points fixed by a line symmetry $L_X$ are precisely the points on the line generated by $X$. In this case, this means that the only points fixed by $L_A$ are the points on $L_A$ and the only points fixed by $L_B$ are the points on $L_B$. Since $L_A$ and $L_B$ are parallel, they do not intersect, so there is no point fixed by both $L_A$ and $L_B$ unless $L_A = L_B$.
        
        Considering the action of line symmetries on lines in $\R^2$, though, we know that the only lines fixed by a line symmetry are the lines perpendicular to the line of reflection or the line of reflection itself. Since $L_A$ and $L_B$ are parallel, the only line fixed by both $L_A$ and $L_B$ is the line perpendicular to $L_A$ and $L_B$.
    \color{black}


\pagebreak
3.  In the following problems, we construct an embedding of the free group of rank 2, $F_{2} \lra \SO(3,\R)$.  Let $\Gamma = \langle A, B \rangle$.  We prove that $A$ and $B$ have no relations.  Denote
\[A = \left(
    \begin{array}{ccc}
    1 & 0 & 0 \\
    0 & 3/5 & -4/5 \\
    0 & 4/5 & 3/5
\end{array}\right) \text{ and } B = \left(
    \begin{array}{ccc}
    3/5 & -4/5 & 0 \\
    4/5 & 3/5 & 0 \\
    0 & 0 & 1
    \end{array}
\right)\]
Prove that $A$ generates an infinite cyclic group. (Hint:  Some basic number theory may be helpful here)

    \color{blue}
        Let $G = \brak{A}$. After reduction of $AA^{-1} = A^{-1}A = I$, we have that all elements $g \in G$ will be of the form $A^n$ for some $n \in \Z$. 
        
        It suffices to show that that for all $n \neq 0 \in \Z$, $A^n \neq I$ since if $A^m = A^n$ for some $m \neq n$, then $A^{m-n} = I$.

        Suppose $A^n = I$.
        
        We proceed by cases: If $n$ is even then $A^n = A^{2m}$ for some $m$. Thus, 
        \[A^m \, A^m = I \implies A^m = A^{-m} \implies m = -m \implies m = 0\]
        This is a contradiction so no even $n$ will satisfy $A^n = I$.

        If $n$ is odd, then $A^n = A^{2m+1}$ for some $m$. Thus,
        \[A^{2m}\, A = I \implies A^{2m} = A^{-1} \implies 2m = -1\]
        but this has no solution in the integers, so no odd $n$ will satisfy $A^n = I$. 
        
        Thus, $A$ generates an infinite cyclic group. $\qed$

    \color{black}

\pagebreak
4.  Convince yourself that any word ending in $A$ in $\Gamma$ may be expressed as $C^{k_{m}}\hdots A^{k_{3}}B^{k_{2}}A^{k_{1}}$ where $k_{i} \neq 0$ and $n \geq 1$ and where $C$ is either $A$ and $B$ depending on the parity of $m$.  Denote such words by $\mathcal{R}_{A}$.  Prove that for each such $g \in \mathcal{R}_{A}$ one has that there exists $i,j,k \in \mathbb{Z}$ such that 
\[ge_{3} = 
 \left(
\frac{i}{5^{N}}, \frac{j}{5^{N}}, \frac{k}{5^{N}}
 \right) \text{ where } N = |k_{m}| + |k_{m-1}| + \dots + |k_{1}|\]

    \color{blue}
        We can write 
        \[A = \frac{1}{5}A_0 = \frac{1}{5}\begin{pmatrix}
            5 & 0 & 0\\
            0 & 3 & -4\\
            0 & 4 & 3
        \end{pmatrix}, \quad B = \frac{1}{5}B_0 = \frac{1}{5}\begin{pmatrix}
            3 & -4 & 0\\
            4 & 3 & 0\\
            0 & 0 & 5
        \end{pmatrix}\]

        Therefore, 
        \begin{align*}
            g &= C^{k_m}\dots A^{k_3}B^{k_2}A^{k_1}\\ 
                &= \frac{C_0^{k_m}}{5^{\abs{k_m}}} \cdots \frac{A_0^{k_3}}{5^{\abs{k_3}}} \cdot \frac{B_0^{k_2}}{5^{\abs{k_2}}} \cdot \frac{A_0^{k_1}}{5^{\abs{k_1}}}\\
                &= \frac{1}{5^{\abs{k_m}} \cdots 5^{\abs{k_2}} \cdot 5^{\abs{k}} \cdot 5^{\abs{k_1}}}\cdot C_0^{k_m}\cdots A_0^{k_3} \cdot B_0^{k_2} \cdot A_0^{k_1}\\
                &= \frac{1}{5^{\abs{k_m} + \cdots + \abs{k_2} + \abs{k_1}}} \cdot C_0^{k_m}\cdots A_0^{k_3} \cdot B_0^{k_2} \cdot A_0^{k_1}\\ 
                &= \frac{1}{5^N} C_0^{k_m}\cdots A_0^{k_3} \cdot B_0^{k_2} \cdot A_0^{k_1}\\ 
                &= \frac{1}{5^N} \begin{pmatrix}
                    * & * & *\\ 
                    * & * & *\\
                    * & * & *
                \end{pmatrix}
        \end{align*}
        where $N = \abs{k_m} + \cdots + \abs{k_2} + \abs{k_1}$.

        We can thus write 
        \[ge_3 = \frac{1}{5^N} \begin{pmatrix}
            *\\*\\*
        \end{pmatrix}\]

        However, since all entries of $A_0$ and $B_0$ are integers, we know the product $C_0^{k_m}\cdots A_0^{k_3} \cdot B_0^{k_2} \cdot A_0^{k_1}$ will have integer entries. Thus, we can more strongly say 
        \[ge_3 = \left(\frac{i}{5^N}, \frac{j}{5^N}, \frac{k}{5^N}\right)\]
        for $i, j, k \in \Z$ and $N = \abs{k_m} + \cdots + \abs{k_2} + \abs{k_1}$. $\qed$
    \color{black}


\pagebreak 

5.  Calculate the following quantities.  
\[Ae_{3},\phantom{=} A^{2}e_{3} \phantom{=} A^{3}e_{3} \phantom{=} BAe_{3} \phantom{=} BA^{2}e_{3} \phantom{=} BA^{3}e_{3} \phantom{=} B^{2}Ae_{3} \phantom{=} B^{2}A^{2}e_{3} \phantom{=}B^{2}A^{3}e_{3}\]
Rationalize each by powers of $5^{N}$ where $N$ is as it was defined in Problem 4.  Consider the second coordinate of each of the above post rationalization, and evaluate it mod 4.  You should notice a pattern.  

Observe that for each such choice of element $g \in \mathcal{R}_{A}$, the second coordinate of $ge_{3}$ is expressible as $(4j)/5^{m}$ where $j \in \mathbb{Z}$ \emph{ and } $j \neq 0 \mod 5$ and $m \geq 1$. 

    \color{blue}
        \begin{align*}
            Ae_3 &= \begin{pmatrix}
                1 & 0 & 0\\ 
                0 & 3/5 & -4/5\\
                0 & 4/5 & 3/5
            \end{pmatrix}\begin{pmatrix}
                0\\0\\1
            \end{pmatrix} = \begin{pmatrix}
                0\\-4/5\\3/5
            \end{pmatrix} = \frac{1}{5} \begin{pmatrix}
                0\\-4\\3
            \end{pmatrix}\\ 
            A^2e_3 &= \begin{pmatrix}
                1 & 0 & 0\\ 
                0 & -7/25 & -24/25\\
                0 & 24/25 & -7/25
            \end{pmatrix}\begin{pmatrix}
                0\\0\\1
            \end{pmatrix} = \begin{pmatrix}
                0\\-24/25\\7/25
            \end{pmatrix} = \frac{1}{5^2} \begin{pmatrix}
                0\\-24\\7
            \end{pmatrix}\\ 
            A^3e_3 &= \begin{pmatrix}
                1 & 0 & 0\\ 
                0 & -117/125 & -44/125\\
                0 & 44/125 & -117/125
            \end{pmatrix}\begin{pmatrix}
                0\\0\\1
            \end{pmatrix} = \begin{pmatrix}
                0\\-44/125\\117/125
            \end{pmatrix} = \frac{1}{5^3} \begin{pmatrix}
                0\\-44\\117
            \end{pmatrix}\\ 
            BAe_3 &= \begin{pmatrix}
                3/5 & -4/5 & 0\\ 
                4/5 & 3/5 & 0\\
                0 & 0 & 1
            \end{pmatrix} \begin{pmatrix}
                1 & 0 & 0\\ 
                0 & 3/5 & -4/5\\
                0 & 4/5 & 3/5
            \end{pmatrix}\begin{pmatrix}
                0\\0\\1
            \end{pmatrix} = \begin{pmatrix}
                3/5 & -12/25 & 16/25\\ 
                4/5 & 9/25 & -12/25\\
                0 & 4/5 & 3/5
            \end{pmatrix}\begin{pmatrix}
                0\\0\\1
            \end{pmatrix} = \begin{pmatrix}
                16/25\\-12/25\\3/5
            \end{pmatrix} = \frac{1}{5^2} \begin{pmatrix}
                16\\-12\\15
            \end{pmatrix}\\ \\
            BA^2e_3 &= \begin{pmatrix}
                3/5 & 28/125 & 96/125\\ 
                4/5 & -21/125 & -72/125\\
                0 & 24/25 & -7/25
            \end{pmatrix}\begin{pmatrix}
                0\\0\\1
            \end{pmatrix} = \begin{pmatrix}
                96/125\\-72/125\\-7/25
            \end{pmatrix} =\frac{1}{5^3} \begin{pmatrix}
                96\\-72\\-35
            \end{pmatrix}\\ \\
            BA^3e_3 &= \begin{pmatrix}
                3/5 & 468/625 & 176/625\\
                4/5 & -351/625 & -132/625\\
                0 & 44/125 & -117/125
            \end{pmatrix}\begin{pmatrix}
                0\\0\\1
            \end{pmatrix} = \begin{pmatrix}
                176/625\\-132/625\\-117/125
            \end{pmatrix} = \frac{1}{5^4}\begin{pmatrix}
                176\\-132\\-585
            \end{pmatrix}\\
            B^2Ae_3 &= \begin{pmatrix}
                -7/25 & -72/125 & 96/125\\
                24/25 & -21/25 & 28/25\\
                0 & 4/5 & 3/5
            \end{pmatrix}\begin{pmatrix}
                0\\0\\1
            \end{pmatrix} = \begin{pmatrix}
                96/125\\28/25\\3/5
            \end{pmatrix} = \frac{1}{5^3}\begin{pmatrix}
                96\\140\\75
            \end{pmatrix}\\
            B^2A^2e_3 &= \begin{pmatrix}
                -7/25 & 168/625 & 576/625\\
                24/25 & 49/625 & 168/625\\
                0 & 24/25 & -7/25
            \end{pmatrix}\begin{pmatrix}
                0\\0\\1
            \end{pmatrix} = \begin{pmatrix}
                576/625\\168/625\\-7/25
            \end{pmatrix} = \frac{1}{5^4}\begin{pmatrix}
                576\\168\\-175
            \end{pmatrix}\\
            B^2A^3e_3 &= \begin{pmatrix}
                -7/25 & 2808/3125 & 1056/3125\\
                24/25 & 819/3125 & 308/3125\\
                0 & 44/125 & -117/125
            \end{pmatrix}\begin{pmatrix}
                0\\0\\1
            \end{pmatrix} = \begin{pmatrix}
                1056/3125\\308/3125\\-117/125
            \end{pmatrix} = \frac{1}{5^5}\begin{pmatrix}
                1056\\308\\ -2925
            \end{pmatrix}
        \end{align*}

        Looking at the second coordinates of each of these vector and reducing mod 4, we have:
        \begin{center}
            \begin{tabular}{|c|c|}
                \hline 
                $g$ & $g e_3 \mod 4$\\
                \hline
                -4 & 0\\ 
                -24 & 0\\ 
                -44 & 0\\
                -12 & 0\\
                -72 & 0\\
                -132 & 0\\
                140 & 0\\
                168 & 0\\
                308 & 0\\ 
                \hline
            \end{tabular}
        \end{center}
        so we conclude the second coordinate of each $ge_3$ is expressible $(4j)/^m$ where $j \in \Z$ with $j \neq 0 \mod 5$ and $m \geq 1$.

    \color{black}


\pagebreak
6.  Assuming the above observation, prove that for each $g \in \mathcal{R}_{A}$, that $g$ is non-trivial.  (By entirely analogous arguments this shows that each element of $g \in \mathcal{R}_{B}$ is non-trivial, so $\Gamma$ is free)

    \color{blue}
        From (4), for each $g \in \mathcal{R}_A$, 
        \[ge_3 = \left(\frac{i}{5^N}, \frac{j}{5^N}, \frac{k}{5^N}\right)\]

        From (5), we have the stronger condition that the second coordinate is expressible as $(4j)/5^m$ with $j \in \Z$ and $j \neq 0 \mod 5$ and $m \geq 1$.

        To show that $g$ is non-trivial, it suffices to show that the second coordinate of $ge_3$ is non-zero.

        Suppose $(4j)/5^m = 0$. Clearly, this suggests 
        \[4j = 0 \implies j = 0 \implies j \mod 5 = 0 \mod 5 = 0\] 
        but this is a contradiction, so the second coordinate of $ge_3$ is non-zero. Therefore, $g$ is non-trivial. $\qed$
    \color{black}


\pagebreak
7.  Finish the proof of the Euler-Rodrigues theorem we mostly finished in class. Namely that for each $q \in S^{3}$, if we express $q = \cos(\theta/2) + u\sin(\theta/2)$ for $\theta \in [0,2\pi]$ and $u \in P\cap S^{3}$, where $P$ is the pure quaternions, then $I_{q} : P \lra P$ is rotation about $u$ by an angle of $\theta$ in the \emph{clockwise} direction.  Finish the proof by showing the rotation is indeed clockwise as claimed.

    \color{blue}
        Let $q = v^{-1}w$ where $v$ and $w$ are pure unit (based on a lemma from class). Since $v, w \in P \cap S^3$, $q = (-v)w$. 

        Further, $I_q = I_{-v} \circ I_w$ (where $I_{-v}$ and $I_w$ are line symmetries generated by $-v$ and $w$ respectively) is a rotation by $\theta = 2\phi$ where $\phi$ is the angle from $w$ to $-v$. So 
        \begin{align*}
            q &= (-v)w\\ 
            &= -(-v \cdot w) + (-v) \times w\\ 
            &= v \cdot w - v \times w
        \end{align*}

        Since $v \cdot w = \abs{v}\, \abs{w} \cos \phi = \cos \phi = \cos \frac{\theta}{2}$, we can say 
        \[q = \cos(\theta/2) - v\times w\]
      
        Since $q \in S^3$, $\norm{q} = 1$ so $\cos^2(\theta/2) + \norm{-v \times w}^2 = 1$. Thus, $\sin^2(\theta/2) = \norm{-v \times w}^2$ so $\sin(\theta/2) = \norm{-v \times w}$.

        Therefore, we can write $q = \cos(\theta/2) + u\sin(\theta/2)$ where $u$ is a unit vector in the direction of $v \times w$ corresponding to a clockwise rotation by $\theta$ (since the frame is right-handed, $-v \times w$ corresponds to a counterclockwise rotation so the rotation would be clockwise relative to $v \times w$). $\qed$
    \color{black}


\pagebreak
8.  Consider the embedding of $\Ha^{\times} \lra \GL(4,\R)$ as described in Homework 7.  Let $\h$ denote the Lie-algebra of $\Ha^{\times} \subset \GL(4,\R)$.  Prove that $\mathfrak{q} = \h/\mathfrak{z}(\h)$ is isomorphic to $\mathfrak{so}(3,\R)$.  

    \color{blue}
        Elements of $\h$ are of the form 
        \[\begin{pmatrix}
            a & -b & -c & -d\\
            b & a & -d & c\\
            c & d & a & -b\\
            d & -c & b & a
        \end{pmatrix}\]

        Elements of $\mathfrak{so}(3, \R)$ are of the form 
        \[\begin{pmatrix}
            0 & -c & b\\
            c & 0 & -a\\
            -b & a & 0
        \end{pmatrix}\]

        Consider the map $\phi: \h \to \mathfrak{so}(3, \R)$ given by
        \[\begin{pmatrix}
            a & -b & -c & -d\\
            b & a & -d & c\\
            c & d & a & -b\\
            d & -c & b & a
        \end{pmatrix} \mapsto \begin{pmatrix}
            0 & -b & d\\
            b & 0 & -c\\
            -d & c & 0
        \end{pmatrix}\]

        We can check that this is a homomorphism:
        \begin{align*}
            \phi\left(\begin{pmatrix}
                a & -b & -c & -d\\
                b & a & -d & c\\
                c & d & a & -b\\
                d & -c & b & a
            \end{pmatrix} + \begin{pmatrix}
                w & -x & -y & -z\\
                x & w & -z & y\\
                y & z & w & -x\\
                z & -y & x & w
            \end{pmatrix}\right) &= \phi\left(\begin{pmatrix}
                a+w & -(b+x) & -(c+y) & -(d+z)\\
                b+x & a+w & -(d+z) & c+y\\
                c+y & d+z & a+w & -(b+x)\\
                d+z & -(c+y) & b+x & a+w
            \end{pmatrix}\right)\\ 
            &= \begin{pmatrix}
                0 & -(b+x) & d+z\\
                b+x & 0 & -(c+y)\\
                -(d+z) & c+y & 0
            \end{pmatrix}\\ 
            \phi\begin{pmatrix}
                a & -b & -c & -d\\
                b & a & -d & c\\
                c & d & a & -b\\
                d & -c & b & a
            \end{pmatrix} + \phi\begin{pmatrix}
                w & -x & -y & -z\\
                x & w & -z & y\\
                y & z & w & -x\\
                z & -y & x & w
            \end{pmatrix} &= \begin{pmatrix}
                0 & -b & d\\
                b & 0 & -c\\
                -d & c & 0
            \end{pmatrix} + \begin{pmatrix}
                0 & -x & z\\
                x & 0 & -y\\
                -z & y & 0
            \end{pmatrix}\\
            &= \begin{pmatrix}
                0 & -b -x & d+z\\
                b+x & 0 & -c-y\\
                -d-z & c+y & 0
            \end{pmatrix}
        \end{align*}

        And clearly it is surjective. 

        Notice that $\ker \phi = \R I_4$. This is precisely $\mathfrak{z}(\h)$ since in HW 7, we found that with basis 
        \[A = \begin{pmatrix}
            1\\ 
            & 1\\ 
            & & 1\\
            & & & 1
        \end{pmatrix}, B = \begin{pmatrix}
            & -1\\ 
            1 \\ 
            & & & -1\\ 
            & & 1
        \end{pmatrix}, C = \begin{pmatrix}
            & & -1\\ 
            & & & 1\\ 
            1\\ 
            & -1
        \end{pmatrix}, D = \begin{pmatrix}
            & & & -1\\ 
            & & -1\\ 
            & 1\\ 
            1
        \end{pmatrix}\]
        the lie algebra of $\h$ is defined by brackets 
        \[[A, B] = 0, [A, C] = 0, [A, D] = 0, [B, C] = 2A, [B, D] = -2C, [C, D] = 2B\]
        so $\mathfrak{z}(\h) = \brak{A} = \R I_4$. 

        Since $\phi$ is a surjective homomorphism $\h \to \mathfrak{so}(3, \R)$ with kernel $\mathfrak(z)(\h)$, by the first isomorphism theorem, 
        \[\h/\mathfrak{z}(\h) \simeq \mathfrak{so}(3, \R) \qed\]
    \color{black}


\pagebreak

\textbf{Bonus: [3 pts]} In the context of the proof above where we construct a free group of rank 2 inside $\SO(3,\R)$, prove that for any $\epsilon > 0$, there exists an $1 \neq g \in \Gamma$ such that $||g - I_{3}|| < \epsilon$.  

\textbf{Bonus: [4 pts]} Construct an element $1 \neq g \in \Gamma$ such that $||g - I_{3}|| < 10^{-4}$.  Prove your bound is sharp.  

\textbf{Bonus: [5 pts]} Provide a rigorous proof that each element $g \in \mathcal{R}_{B}$ is not-trivial.  


\end{document}