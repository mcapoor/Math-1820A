\documentclass[11pt]{article}

%%%%%%%%%%%%%%%%%%%%%%
%%%%%%% PACAKAGES %%%%%%%
%%%%%%%%%%%%%%%%%%%%%


%%%%%%%%%%%%% GRAPHICS/FONTS
\usepackage{amsmath,amsfonts,amssymb,graphicx,tikz-cd,pgfplots}

%%%%%%%%%%%%%% FORMATTING
\usepackage{geometry,titlesec,hyperref,xhfill,float,fancyhdr}
\pagestyle{fancy}
\usepackage[onehalfspacing]{setspace}

%%%%%%%%DEFINING COMMANDS
\usepackage{xifthen}

%%%%%%%%%TIKZ STUFF
\usetikzlibrary{positioning,calc}
\usetikzlibrary{decorations.markings}
\usepgfplotslibrary{polar}
\usepgflibrary{shapes.geometric}
\usepgfplotslibrary{fillbetween}

%%%%%%FOR THE GRAPHS
\pgfplotsset{my style/.append style={axis x line=middle, axis y line=
middle, xlabel={$x$}, ylabel={$y$}, axis equal }}
\pgfplotsset{compat=1.18}

%%%%%%MARGINS
\renewcommand{\headrulewidth}{0pt} 
\geometry{
  letterpaper,
  left=0.5in,
  right=0.5in,
  top=0.5in,
  bottom=0.5in
}
\setlength{\parskip}{0pt}
\setlength{\parindent}{0pt}

%%%%%%MAKES THE HEADER AND FOOTER FANCY
\fancyhf{}
\renewcommand{\headrulewidth}{0pt}
\pagestyle{fancy}

%%%%%%%%%%%%%%%%%%%%%%
%%%%%%% COMMANDS %%%%%%%
%%%%%%%%%%%%%%%%%%%%%
\newcommand{\underscore}{\underline{\hspace{2mm}}}
\newcommand{\Z}{\mathbb{Z}}
\newcommand{\N}{\mathbb{N}}
\newcommand{\Q}{\mathbb{Q}}
\newcommand{\C}{\mathbb{C}}
\newcommand{\R}{\mathbb{R}}
\newcommand{\Aut}{\text{Aut}}
\newcommand{\GL}{\text{GL}}
\newcommand{\SO}{\text{SO}}
\newcommand{\PGL}{\text{PGL}}
\newcommand{\Stab}{\text{Stab}}
\newcommand{\End}{\text{End}}
\newcommand{\lra}{\longrightarrow}

\newcommand{\RP}{\mathbb{R}P}
\newcommand{\K}{\emph{K}}

\newcommand{\ihat}{\hat{\textbf{i}}}
\newcommand{\jhat}{\hat{\textbf{j}}}
\newcommand{\khat}{\hat{\textbf{k}}}

\newcommand{\gf}{\mathfrak{g}}
\newcommand{\tr}{\text{tr}\,}
\newcommand{\qed}{\quad \blacksquare}

%TEXT COMMAND
\newcommand{\T}[1][]{\text{#1}}
\newcommand{\TB}[1][]{\mathbb{#1}}

\newcommand{\xlra}[1][]{%
  \ifthenelse{\isempty{#1}}%
    {\xrightarrow{\phantom{,,,,,,}}}% if #1 is empty
    {\xrightarrow{\phantom{,,}#1\phantom{,,}}}% if #1 is not empty
}


%%%%%%%%%%%%%%%%%%%%%%%%%%%%%%%%%%%%%%%%
%%%%%%%%BEGINNING OF ACTUAL DOCUMENT %%%%%%%%%%
%%%%%%%%%%%%%%%%%%%%%%%%%%%%%%%%%%%%%%%%



%%%%TITLE______REMEMBER TO REGULARLY CHANGE THIS!!!!
\title{Math 1820A Spring 2024 - Homework 3}
\date{}

\begin{document}
\maketitle
\vspace{-0.5in}
%%%%%%%%%%%%%%%%

\noindent \textbf{Instructions:}  This assignment is worth twenty points.  Please complete the following problems assigned below.  Submissions with insufficient explanation may lose points due to a lack of reasoning or clarity.  If you are handwriting your work, please ensure it is readable and well-formatted for the grader.

Be sure when uploading your work to \textbf{assign problems to pages}.  Problems with pages not assigned to them \textbf{may not be graded}.  


%%%%%%%%%%%%%%%%%%
\vspace{10mm}\noindent
\textbf{Textbook Problems: }  
\vspace{10mm}\noindent

\noindent
\textbf{Additional Problems:}   For these problems if you see an $S$ in front of a group, you can assume it means determinant 1, e.g. all elements of $SO(3,\R)$ and $SO(2,1)$ have determinant 1.
\vspace{10mm}\noindent

1.  Recall in class we defined the Lie-algebra of a group $G \subset \GL(n,\C)$ to be 
\[\mathfrak{g} = \{B \in M_{n}(\C) \, | \, B = \gamma'(0) \text{ where }\gamma \text{ is a smooth path in }G\text{ satisfying }\gamma(0) = 1\}
\]
Prove there is a one-to-one correspondence between one-parameter subgroups of $G$ and $\mathfrak{g}$.  Note this says that every one-parameter subgroup is in fact exponential.  \\

	\color{blue}
		Let $A(t)$ be a one-parameter subgroup of $G$. By definition, it satisfies 
		\begin{enumerate}
			\item $A$ is continuous 
			\item $A(0) = 1$
			\item $A(t+s) = A(t)A(s)$
		\end{enumerate}

		By Hall Th. 2.13, there exists a unique $n \times n$ complex matrix $X$ such that 
		\[A(t) = e^{tX}\]

		Differentiating with respect to $t$ and evaluating at $t=0$, we have 
		\[A'(0) = Xe^{(0)X} = X\]
		i.e. $X \in M_n(\C)$ and $X = A'(0)$ where $A$ is a smooth path in $G$ satisfying $A(0) = 1$, so $X \in \gf$.  

		By the uniqueness of $X$, we have a one-to-one correspondence between one-parameter subgroups of $G$ and $\gf$. $\qed$
	\color{black}

\pagebreak

2.  Explain briefly why if we have a smooth path $\gamma : \R \to G$ so that $\gamma(0) = 1$, that we may approximate $\gamma$ by a one-parameter subgroup up to first-order.  Note this means that we may define 
\[\mathfrak{g} = \{B \in M_{n}(\C) \, | \, \gamma(t) = e^{tB} \text{ is a smooth path in }G\}\]

	\color{blue}
		Since it is smooth and $\gamma(0) = 1$, we may approximate $\gamma$ by its first order Maclaurin polynomial 
		\[\gamma(t) \approx \gamma(0) + t\gamma'(0) = 1 + t\gamma'(0)\] 

		Notice, however, that up to first order, this is exactly $\exp(t\gamma'(0))$, a one parameter subgroup. Therefore, $e^{tB}$ where $B = \gamma'(0)$ is a good approximation for $\gamma$. 
	\color{black}

\pagebreak	

3.  Let $G = \SO(2,1)$ be the group of determinant 1 matrices in $\GL(3,\R)$ such that $(Av,Aw) = (v,w)$ where $(v,w)$ is the Lorentzian product with signature $(+,+,-)$.  Calculate $\mathfrak{g}$, and find a basis for $\mathfrak{g}$.  Write out an interesting exponential to get a neat one-parameter family in $G$.\\
	
	\color{blue}
		From Hall 2.5.6, a matrix $A$ is in $O(2,1)$ if and only if $A^{T}gA = g$ (or equivalently, $g^{-1}A^Tg= A^{-1}$) where 
		\[g = \begin{pmatrix}
			1 & 0 & 0\\
			0 & 1 & 0\\
			0 & 0 & -1
		\end{pmatrix}\]

		Then for $X$ a $3\times 3$ matrix, we have $X \in \gf$ if and only if $\exp(tX) \in \SO(2,1)$, i.e. 
		\[g^{-1}e^{tX^T}g = A^{-1}\]
		notice, however, that 
		\[g^2 = I \implies g = g^{-1}\]
		so we have 
		\[g^{-1}e^{tX^T}g = ge^{tX^T}g = e^{tgX^Tg} = e^{-tX}\]
		This condition holds for all $t$ if $gX^Tg = -X$.

		Adding the condition that $\det(A) = 1$, we have $\tr X = 0$. Thus, 
		\[\gf = \{X \in M_3(\R) \; | \; gX^Tg = -X \text{ and } \tr X = 0\}\]

		These are matrices of the form 
		\[\begin{pmatrix}
			0 & -a & b\\
			a & 0 & c\\
			b & c & 0
		\end{pmatrix}\]
		so we can introduce the basis 
		\[A = \begin{pmatrix}
			0 & -1 & 0\\
			1 & 0 & 0\\
			0 & 0 & 0
		\end{pmatrix}, \quad B = \begin{pmatrix}
			0 & 0 & 1\\
			0 & 0 & 0\\
			1 & 0 & 0
		\end{pmatrix}, \quad C = \begin{pmatrix}
			0 & 0 & 0\\
			0 & 0 & 1\\
			0 & 1 & 0
		\end{pmatrix}\]
		and then find that 
		\begin{align*}
			[A, B] &= C\\
			[A, C] &= -B\\
			[B, C] &= -A
		\end{align*}

		Then notice that 
		\begin{align*}
			\exp(tA) &= I + tA + \frac{1}{2}t^2A^2 - \frac{1}{3!}t^3A - \frac{1}{4!}t^4A^2 + \frac{1}{5!}t^5A + \frac{1}{6!}t^6A^2 + \dots\\ 
				&= A\sin(t) - A^2\cos(t)\\ 
				&= \begin{pmatrix}
					\cos(t) & -\sin(t) & 0\\
					\sin(t) & \cos(t) & 0\\
					0 & 0 & 1
				\end{pmatrix}
		\end{align*} 
	\color{black}

\textbf{Bonus: [3 pts]}  In the context of Problem 3, prove that a vector $v \in T_{p}H^{+}$ if and only if $(v,p) = 0$. 
\pagebreak

4.  Prove that the linear action of $G$ on $\R^{3}$ in Problem 3 preserves the set $S = \{p \in \R^{3} \, | \, (p,p) = -1\}$.  Let $H = \Stab(p)$ where $p = (0,0,1)$.  Calculate a basis for its Lie-algebra $\mathfrak{h}$.\\

	\color{blue}
		Let $p = (x, y, z) \in \R^3$. Then the set of points $S$ is given by the set of points for whom 
		\[(p, p) = x^2 + y^2 - z^2 = -1\]

		From calculus, this is a hyperboloid of two sheets. Problem 3 shows that the action of $G$ is rotations in 3-space. Clearly, the linear action of $G$ preserves $S$ since the hyperboloid is symmetric about the $z$-axis. 

		In fact, $H$ is given by precisely the one-parameter family from Problem 3: 
		\[\begin{pmatrix}
			\cos(t) & -\sin(t) & 0\\
			\sin(t) & \cos(t) & 0\\
			0 & 0 & 1
		\end{pmatrix} \begin{pmatrix}
			0\\
			0\\
			1
		\end{pmatrix} = \begin{pmatrix}
			0\\ 
			0\\
			1
		\end{pmatrix}\]
		
		In problem 3, we showed that $\exp(tA) = \begin{pmatrix}
			\cos(t) & -\sin(t) & 0\\
			\sin(t) & \cos(t) & 0\\
			0 & 0 & 1
		\end{pmatrix}$ for $A = \begin{pmatrix}
			0 & -1 & 0\\
			1 & 0 & 0\\
			0 & 0 & 0
		\end{pmatrix}$. Since $H$ is a one parameter family, a basis for $\mathfrak{h}$ is given simply by $\{A\}$.

	\color{black}

\pagebreak
5.  Recall the construction of a semi-direct product.  Let $N$ and $H$ be groups, and let $\phi:  H \lra\Aut(N)$ be a group homomorphism.  Form the semi-direct product $G = N\rtimes_{\phi} H$ to the following.  As a \emph{set}, $G$ is simply $N \times H$.  As a \emph{group}, the group law is
\[(n,h)(a,b) := (n\phi_{h}(a),hb)\]

Prove that $N$ is a normal subgroup of $G$, and that $G/N \simeq H$.  In fact, show that there exists a \emph{section} $q$ in the short exact sequence
\[N \xrightarrow{i} G \xrightarrow{q} H\]

That is, show there exists a $\sigma : H \lra G$ such that $q\sigma = 1_{H}$.  \\

	\color{blue}
		$N$ is normal in $G$ iff for all $n \in N$ and $g \in G$, $gng^{-1} \in N$. Let $n \in N$ and $g = (n', h) \in G$. 

		First, we need to embed $n \mapsto (n, 1) \in G$. Then notice that the inverse of an element $(a, b) \in G$ is given by $(\phi_b^{-1}(a^{-1}), b^{-1})$: 
		\[(a, b)(\phi_b^{-1}(a^{-1}), b^{-1}) = (a\phi_b(\phi_b^{-1}(a^{-1})), bb^{-1}) = (aa^{-1}, bb^{-1}) = (a, b)\]

		Now taking conjugates, let $n \in N$ and $(a, b) \in G$ so 
		\begin{align*}
			(a, b)(n, 1)(a, b)^{-1} &= (a, b)(n, 1)(\phi_b^{-1}(a^{-1}), b^{-1})\\ 
				&= (a\phi_b(n), b)(a\phi_b^{-1}(a^{-1}), b^{-1})\\
				&= (a\phi_b(n) \cdot \phi_b(\phi_b^{-1}(a^{-1})), bb^{-1})\\
				&= (a\phi_b(n)a^{-1}, 1)\\
		\end{align*}
		Now we just take the inverse $(n, h) \mapsto n$ of the initial embedding to get $a\phi_b(n)a^{-1}\in N$. 
			
		Since $a \in N$, clearly conjugation is in $N$ and thus $N$ is normal in $G$. $\qed$ 
	\color{black}

\pagebreak

6.  Let $\phi: \R \lra \Aut(\R^{2})$ be a homomorphism, and form $G_{\phi} := \R^{2} \rtimes_{\phi} \R$.  For each $v \in \R^{2}$, and $(n,h) \in G$, define $(n,h)v:= \phi(h)v + n$.  Prove this defines a group action of $G_{\phi}$ on $\R^{2}$.\\

	\color{blue}
		To be a group action, $(n, h) \cdot v$ must satisfy  
		\begin{enumerate}
			\item $(e_{\R^2}, e_\R)\cdot v = v$ 
			\item $(n, h) \cdot ((m, k) \cdot v) = ((n, h)(m, k)) \cdot v \quad$ for $(m, k) \in G$
		\end{enumerate}

		For the first condition,
		\[(e_{\R^2}, e_{\R}) \cdot v = \phi(e)v + e_{\R^2}\]
		since homomorphisms preserve identity, $\phi_{\R}(e) = e$, so 
		\[(e_{\R^2}, e_{\R}) \cdot v = v + (0, 0) = v\] 

		For the second condition, 
		\begin{align*}
			(n, h) \cdot ((m, k) \cdot v) &= (n, h) \cdot (\phi(k)v + m)\\ 
				&= \phi(h)(\phi(k)v + m) + n\\ 
				&= \phi(hk)v + \phi(h)m + n
		\end{align*}
		and 
		\begin{align*}
			((n, h)(m, k)) \cdot v &= (n + \phi(h)m, h + k) \cdot v\\ 
				&= \phi(h + k)v + n + \phi(h)m \\ 
				&= \phi(hk)v + \phi(h)m + n \qed
		\end{align*}

	\color{black}


\pagebreak 

7.  Let $\phi: \R \lra \Aut(\R^{2})$ be defined by 
\[\phi(h) = \begin{pmatrix}
  \cos(h) &-\sin(h)\\
  \sin(h) &\cos(h)
\end{pmatrix}\]
Prove the group $G_{\phi}$ is isomorphic to 
\[
G=
\left\{
\left(\begin{array}{ccc}
\cos(h) &-\sin(h) & a\\ 
\sin(h) &\cos(h) & b\\
0 & 0 & 1
\end{array}
\right) \bigg| \, h,a,b \in \R
\right\} = \text{Isom}^{+}(\R^{2})
\]

	\color{blue}
		Elements of $G_{\phi}$ are of the form $(\begin{pmatrix}
			a\\b
		\end{pmatrix}, c)$ where $a, b, c \in \R$. A natural choice of homomorphism $\psi: G_{\phi} \to G$ is 
		\[\psi((\begin{pmatrix}
			a\\b
		\end{pmatrix}, c)) = \begin{pmatrix} 
			\cos(c) & -\sin(c) & a\\
			\sin(c) & \cos(c) & b\\
			0 & 0 & 1
		\end{pmatrix}\]

		We need to show that $\psi$ is a homomorphism and that it is bijective.

		\begin{align*}
			\psi((\begin{pmatrix}
				a\\b
				\end{pmatrix}, c)(\begin{pmatrix}
				d\\e
				\end{pmatrix}), f) &= \psi((\begin{pmatrix} 
					a\\b
				\end{pmatrix} + \phi_c(\begin{pmatrix} d\\e \end{pmatrix}), c+f))\\ 
					&= \psi((\begin{pmatrix}
						a\\b 
					\end{pmatrix} + \begin{pmatrix}
						\cos(c)d - \sin(c)e\\
						\sin(c)d + \cos(c)e
					\end{pmatrix}, c+f))\\
					&= \psi((\begin{pmatrix}
						a + \cos(c)d - \sin(c)e\\
						b + \sin(c)d + \cos(c)e
					\end{pmatrix}, c+f))\\
					&= \begin{pmatrix}
						\cos(c+f) & -\sin(c+f) & a + \cos(c)d - \sin(c)e\\
						\sin(c+f) & \cos(c+f) & b + \sin(c)d + \cos(c)e\\
						0 & 0 & 1
					\end{pmatrix}\\
			\psi((\begin{pmatrix}
				a\\b \end{pmatrix}), c) \psi((\begin{pmatrix}
				d\\e
				\end{pmatrix}, f)) &= \begin{pmatrix}
					\cos(c) & -\sin(c) & a\\
					\sin(c) & \cos(c) & b\\
					0 & 0 & 1
				\end{pmatrix}\begin{pmatrix}
					\cos(f) & -\sin(f) & d\\
					\sin(f) & \cos(f) & e\\
					0 & 0 & 1
				\end{pmatrix}\\
				&= \begin{pmatrix}
					\cos(c)\cos(f) - \sin(c)\sin(f) & -\cos(c)\sin(f) - \sin(c)\cos(f) & a + d\cos(c) - e\sin(c) \\
					\cos(c)\sin(f) + \cos(f)\sin(c) & -\sin(c)\sin(f) + \cos(c)\cos(f) & b + e\cos(c) + d\sin(c)\\
					0 & 0 & 1
				\end{pmatrix}\\
				&= \begin{pmatrix}
					\cos(c + f) & -\sin(c + f) & a + d\cos(c) - e\sin(c)\\
					\sin(c + f) & \cos(c + f) & b + e\cos(c) + d\sin(c)\\
				\end{pmatrix}
		\end{align*}

		Thus, $\psi$ is a homomorphism. To see that it is a bijection, consider the mapping $\psi^{-1}: G \to G_{\phi}$ given by
		\[\begin{pmatrix}
			\cos(c) & -\sin(c) & a\\
			\sin(c) & \cos(c) & b\\
			0 & 0 & 1
		\end{pmatrix} \mapsto (\begin{pmatrix}
			a\\b
		\end{pmatrix}, c)\]

		So 
		\begin{gather*}
			\psi^{-1}(\psi((\begin{pmatrix}
				a\\b \end{pmatrix}, c))) = \psi^{-1}\begin{pmatrix}
					\cos(c) & -\sin(c) & a\\
					\sin(c) & \cos(c) & b\\
					0 & 0 & 1
				\end{pmatrix} = (\begin{pmatrix}
					a\\b
				\end{pmatrix}, c)\\
			\psi(\psi^{-1}\begin{pmatrix}
				\cos(c) & -\sin(c) & a\\
				\sin(c) & \cos(c) & b\\
				0 & 0 & 1
			\end{pmatrix}) = \psi((\begin{pmatrix}
				a\\b
			\end{pmatrix}, c)) = \begin{pmatrix}
				\cos(c) & -\sin(c) & a\\
				\sin(c) & \cos(c) & b\\
				0 & 0 & 1
			\end{pmatrix}
		\end{gather*}

		Thus, $\psi$ is a bijective homomorphism and thus an isomorphism. $\qed$

	\color{black}

\pagebreak
8.  Calculate the Lie-algebra of the group $G$ as defined in Problem 7.  Show that there is a split-exact sequence of Lie-algebras
\[
\R^{2} \lra \mathfrak{g} \lra \R
\]  

	\color{blue}
		Let $X \in \gf$. Then $e^{tX}$ is of the form 
		\[\begin{pmatrix}
			\cos(t) & -\sin(t) & a\\
			\sin(t) & \cos(t) & b\\
			0 & 0 & 1	
		\end{pmatrix} = \begin{pmatrix}
			R & \begin{matrix}
				a\\b
			\end{matrix}\\ 
			\begin{matrix}
				0 & 0
			\end{matrix} & 1
		\end{pmatrix}\] 
		with $R \in \SO(2, \R)$. 

		So $X = \frac{d}{dt} \bigg\vert_{t=0}e^{tX}$ must have the form 
		\[\begin{pmatrix}
			Y & \begin{matrix}
				y_1\\y_2
			\end{matrix}\\ 
			\begin{matrix}
				0 & 0
			\end{matrix} & 0
		\end{pmatrix}\] 

		Notice that for $n \geq 1$, 
		\[\begin{pmatrix}
			Y & \begin{matrix}
				y_1\\y_2
			\end{matrix}\\ 
			\begin{matrix}
				0 & 0
			\end{matrix} & 0
		\end{pmatrix}^n = \begin{pmatrix}
			Y^n & \rule{0.6cm}{0.2mm} & Y^{n-1}y \\
			\vline & & \vline\\
			0 & 0 & 0
		\end{pmatrix}\]
		where $y = (y_1, y_2)^T$

		Thus, $e^{tX}$ is of the form 
		\[\begin{pmatrix}
			e^{tY} & \begin{matrix}
				*\\
				*
			\end{matrix}\\
			\begin{matrix}
				0 & 0
			\end{matrix} & 1
		\end{pmatrix}\]

		Since we need $e^{tY} \in O(2, \R)$, we need $Y^T = -Y$. All together, 
		\[\gf = \left\{\begin{pmatrix}
			Y & \begin{matrix}
				a\\ b
			\end{matrix}\\ 
			\begin{matrix}
				0 & 0
			\end{matrix} & 0
		\end{pmatrix} \in M_3(\R) \bigg\vert \; Y^T = -Y,\;  a, b \in \R\right\}\]

		From Problem 5, $\R^2 \trianglelefteq G \simeq \R^2 \rtimes_{\phi} \R$. Therefore, we have a short exact sequence of groups 
		\[\R^2 \hookrightarrow G \twoheadrightarrow G/\R^2 = \R\]
		Then on the level of lie-algebras, we have a short exact sequence 
		\[\R^2 \underset{i}{\hookrightarrow} \gf \underset{p}{\twoheadrightarrow} \R\]
		(note that this is a slight abuse of notation as the first equation refers to $\R^2, \R$ as groups and the second refers to them as lie-algebras with trivial brackets). 

		To show the sequence is split exact, we must show that it admits a a section $\sigma: \R \to \gf$ such that $p\sigma = 1$.

		Arbitrarily, we can define 
		\[p\begin{pmatrix}
			Y & \begin{matrix}
				a\\ b
			\end{matrix}\\ 
			\begin{matrix}
				0 & 0
			\end{matrix} & 0
		\end{pmatrix} = \det Y\]  
		and 
		\[\sigma(t) = \begin{pmatrix}
			\cos(t) & -\sin(t) & 0\\
			\sin(t) & \cos(t) & 0\\
			0 & 0 & 0
		\end{pmatrix}\] 

		Then for $t \in \R$,
		\[p(\sigma(t)) = p\begin{pmatrix}
			\cos(t) & -\sin(t) & 0\\
			\sin(t) & \cos(t) & 0\\
			0 & 0 & 0
		\end{pmatrix} = \det \begin{pmatrix}
			\cos(t) & -\sin(t)\\
			\sin(t) & \cos(t)
		\end{pmatrix} = 1 \qed\]
	\color{black}
\end{document}